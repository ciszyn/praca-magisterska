\documentclass{article}
\usepackage{graphicx}
\usepackage{pdfpages}
\usepackage[utf8]{inputenc}                                      
\usepackage{csquotes}
\usepackage[T1]{fontenc} 
\usepackage{amsmath,amsfonts,amssymb,amsthm}
\usepackage{graphicx} 
\usepackage{hyperref}
\usepackage[page]{appendix} 
\usepackage{booktabs}
\usepackage{hyperref}
\usepackage[backend=bibtex,maxcitenames=10,mincitenames=2]{biblatex}
\usepackage{indentfirst}
\usepackage{setspace}
\usepackage{float}
\onehalfspacing{}

\usepackage[polish,british]{babel} 
\usepackage{lmodern}
\usepackage{graphicx} 
\usepackage{booktabs}
\frenchspacing
\tolerance=1
\emergencystretch=\maxdimen{}

\nocite{*}
\addbibresource{bibliography}


% \fancypagestyle{phdthesis}{
% \fancyhead[C]{\thepage}
% }

\hyphenpenalty=100000
\widowpenalty=100000
\clubpenalty=100000

\begin{document}

% \justifying

\begin{titlepage}
    \includepdf{titlepage.pdf}
\end{titlepage}

\tableofcontents
\pagebreak

\section*{Abstrakt}

\noindent
{\bf Słowa kluczowe:} Uczenie maszynowe, Uczenie Głębokie, Sieci neuronowe,
ERA5, Prognozowanie pogody, Sztuczna inteligencja
\pagebreak

% Provide an overview of the topic, 
% motivation for the research, and the scope of the study.

% The introduction to your thesis is an essential 
% part of your work, as it provides the reader with an 
% overview of your research, its significance, and its objectives. 
% Here are some important elements to consider including in your introduction:

% Overall, your introduction should provide the reader with a clear 
% understanding of the research problem you are addressing, the methods 
% you have used to explore this problem, and the significance of your 
% findings. It should also engage the reader and motivate them to read 
% on, by highlighting the importance and relevance of your research.

% "The results of this research demonstrate that the application 
% of artificial intelligence and machine learning techniques in 
% weather prediction is a promising approach, as the results 
% obtained are comparable to those achieved by established techniques in the field."

% In this thesis statement, you highlight the contribution 
% of your research in showing the promise of artificial 
% intelligence and machine learning techniques in weather 
% prediction. You also emphasize that your results are 
% comparable to those obtained by other approaches, which 
% underscores the reliability and validity of your research findings.
\section{Wstęp}

Sztuczna inteligencja (AI) jako gwałtownie rozwijająca się 
dziedziną informatyki znajduję coraz więcej zastosowań w 
problemach modelujących zachowanie środowiska które nas otacza. 
Nowe algorytmy oraz metody stosowane w zakresie AI mają coraz
większy wpływ na społeczeństwo oraz na sposób w jaki interpretujemy,
przetwarzamy i wykorzystujemy dostępne dane. Wykorzystanie metod
sztucznej inteligencji pozwala na przeanalizowanie dużych ilości
danych, oraz na odnajdowanie i wzięcie pod uwagę skomplikowanych związków
i informacji w nich ukrytych. Możliwość zastosowania metod AI względem 
szerokiego zakresu typu danych, oraz wszechstronność algorytmów jest jednym z 
powodów które umożliwiają zastosowanie ich w celach prognozowania oraz modelowania
pogody. Rosnące zapotrzebowania na dokładne oraz długoterminowe przewidywanie
w zakresie monitorowania klimatu, wieloletnich oraz krótkoterminowych prognoz
pogody skłania do wykorzystywania coraz to nowych algorytmów i podejść
do problemu przewidywania pogody. Wykorzystanie AI może skutkować jednocześnie 
zwiększeniem efektywności w wykorzystaniu zasobów obliczeniowych, zmniejszeniem
ilości wkładu ludzkiego oraz zwiększeniem jakości prognoz.

Jedną z metod która szczególnie w ostatnich latach zyskała popularność w zakresie
AI jest uczenie głębokie (DL). Zastosowanie DL poprzez uczenie sieci neuronowych
znajduje szerokie zastosowanie w rozpoznawaniu i procesowaniu obrazów, wideo oraz
mowy. Wykorzystaniu propagacji wstecznej w algorytmie uczenia sieci neuronowych
pozwala na szybkie i mniej obciążające stworzenie modelu, przynajmniej w porównaniu
z tradycyjnymi metodami numerycznymi wykorzystywanymi w celach przewidywaniu pogody.

Rozwijające się zbiory danych opisujące parametry opisujące pogodę 
pozwalają na wykorzystywanie algorytmów i metod z zakresu uczenia maszynowego (ML)
oraz AI. W ostatnich latach można zaobserwować wzrost wykorzystania tych metod
przez organizacje klimatyczne oraz coraz więcej badań związanych z wykorzystaniem
AI w numerycznym prognozowaniu pogody, jak i zupełnym zastąpieniu numerycznych
prognoz pogodowych (NWP) poprzez 
AI.

% Background and context: Begin by providing some background 
% information on the topic of weather prediction and the application 
% of artificial intelligence and machine learning techniques in this area. 
% This could include a brief overview of the current state of the field, the 
% challenges that exist, and the potential benefits of using these techniques.
\subsection{Kontekst i zastosowania}

Monitorowanie i przewidywanie pogody znajduje zastosowanie w dziedzinach takich
jak agrokultura, kontrolowanie stanu zanieczyszczeń powietrza, 
przewidywanie warunków drogowych w celu zwiększenia bezpieczeństwa drogowego,
przewidywanie rozprzestrzeniania się pożarów naturalnych,
przewidywanie katastrof naturalnych, zastosowanie odnawialnych źródeł energi,
oraz wiele innych. Dokładne przewidywanie warunków pogodowych ma krytyczne znaczenie
dla prawidłowego funkcjonowania wielu organizacji. Amerykańskie centra
informacji o środowisku szacują wielkość zniszczeń wynikających z działania
pagody w 2015 roku na 7.9 miliardów dolarów\cite{using-artificial-intelligence-to-improve}.
Z drugiej strony możliwości prognozowania nasłonecznienia dają szansę na zaoszczędzenie
455 milionów dolarów do 2040 w ramach wykorzystania odnawialnych źródeł energii.

Porozumienie paryskie jest prawnie wiążącą umową międzynarodową,
której celem jest ograniczenie globalnego ocieplenia o 1.5$^{\circ}$C
do 2050 roku w porównaniu do poziomów przedindustrialnych. Wykorzystanie
uczenia maszynowego mogłoby pozwolić na bardziej przemyślane zarządzanie
gospodarką energetyczną oraz wykorzystanie źródeł energii odnawialnych.

Dotychczas najczęściej stosowanym podejściem do celów prognozowania
pogody są algorytmy numeryczne bazujące na rozwiązywaniu równań różniczkowych
charakteryzujących zachowanie atmosfery i pogody. Skomplikowana natura
problemu oraz fakt że równania opisujące dynamikę cieczy oraz termodynamikę
cechują się chaotycznością sprawiają że numeryczne symulacje komputerowe
nie zawsze osiągają wymaganą dokładność, a ich złożoność obliczeniowa
wymaga dużych klastrów komputerowych w celu osiągnięcia celu.

Chociaż duże sieci neuronowe wykorzystywane do prognozowania pogody
mogą zawierać kilka milionów parametrów, co jest porównywalne z modelami
NWP\cite{can-dl-beat-numerical}, to wykorzystanie takiego modelu już po traningu
jest mało wymagająca obliczeniowo i można się spodziewać o wiele mniejszych wymagań
obliczeniowych do wykorzystywania takiego modelu. Co więcej, maksymalna rozdzielczość
modeli NWP stosowanych do globalnych prognoz nie przekracza 5km, 
co może skutkować pominięciem lokalnych zjawisk. Algorytmy AI dają możliwość
parametryzowania i brania pod uwagę zjawisk o charakterze lokalnym.

W ostatnich latach zaobserwowano wzrastającą tendencę do stosowania hybrydowych
algorytmów bazujących na modelach numerycznych, ale także uczeniu maszynowym
wspomagającym w inicjalizacji modelu, parametryzacji modelu, korygowaniu
wyników generowanych przez model oraz szacowaniu pewności wygenerowanej
predykcji. Zastosowanie modeli bazujących czysto na wykorzystywaniu uczenia
maszynowego dotychczas wykazuje gorszą dokładność od podejść hybrydowych.
Największym problemem w tworzeniu modeli AI w celu predykcji pogody jest
zachowanie ograniczeń fizycznych, które nie są bezpośrednio częścią modelu,
w przeciwieństwie do algorytmów NWP.

Wykorzystanie uczenia głębokiego wiążę się z możliwością zastosowania 
zaawansowanych algorytmów które znalazły już zastosowanie w innych dziedzinach.
Sieci neuronowe umożliwiłiby wykorzystanie Transfer Learning, umożliwiający
wydobycie ze zbioru danych podstawowych relacji i charakterystyk,


% Problem statement: Clearly articulate the research problem you are 
% addressing, and explain why it is important. This could involve discussing 
% the limitations of current weather prediction methods, or the potential 
% benefits of improving the accuracy and reliability of these predictions.
\subsection{Sformuowanie problemu}

% Research objectives: Clearly state the objectives of your research, 
% including any hypotheses you are testing or research questions you 
% are exploring. This helps the reader understand the scope and focus of your work.
\subsection{Cele pracy}

% Research methods: Provide an overview of the research methods you have used, 
% including the data sources you have used, the models you have developed, 
% and the evaluation metrics you have used to measure performance.
\subsection{Metody badania}

% Contribution to the field: Conclude by discussing the contribution 
% of your research to the field of weather prediction and the broader 
% field of artificial intelligence and machine learning. This could 
% include highlighting the novelty of your approach, the potential 
% impact of your findings, or the implications for future research.
\subsection{Wkład w dziedzinę}
% Conduct a comprehensive review of the existing literature 
% on artificial intelligence and machine learning techniques 
% used in weather prediction. Discuss the strengths and limitations 
% of these approaches

% The literature review section of your thesis is a critical component 
% of your research, as it provides a comprehensive overview of the 
% current state of knowledge on the topic of weather prediction and 
% the application of artificial intelligence and machine learning 
% techniques in this area. Here are some important elements to consider 
% including in your literature review:

% Background information: Begin by providing some background information on 
% the topic of weather prediction and the challenges that exist in this area. 
% This could include discussing the impact of weather on human life and the 
% economy, the limitations of current weather prediction methods, and the potential 
% benefits of using artificial intelligence and machine learning techniques to 
% improve accuracy.

% Overview of existing literature: Provide a comprehensive overview of the 
% existing literature on weather prediction and the use of artificial intelligence 
% and machine learning techniques in this area. This could involve summarizing the 
% findings of previous studies, identifying key themes and trends, and highlighting 
% any gaps or limitations in the current literature.

% Theoretical framework: Identify and discuss any theoretical frameworks that are 
% relevant to your research, and explain how these frameworks have been applied in 
% previous studies.

% Methodological approaches: Discuss the different methodological approaches 
% that have been used in previous studies, including the types of data sources used, 
% the modeling techniques employed, and the evaluation metrics used to measure 
% performance.

% Critical analysis: Provide a critical analysis of the existing literature, 
% highlighting any strengths or weaknesses in previous studies, and identifying 
% areas where further research is needed.

% The contribution of your research: Conclude by discussing how your research 
% contributes to the current state of knowledge on weather prediction and the 
% application of artificial intelligence and machine learning techniques in this area. 
% This could involve discussing the novelty of your approach, the potential 
% impact of your findings, or the implications for future research.

% Overall, your literature review should provide a comprehensive overview of 
% the existing literature on weather prediction and the application of artificial 
% intelligence and machine learning techniques in this area. It should also 
% highlight the importance of your research question, and explain how your research 
% contributes to the broader understanding of the field.
\section{Przegląd literatury}
% Describe the methodology adopted for the research, 
% including the data collection methods, techniques used 
% for data preprocessing, and the selection of the algorithms 
% used for weather prediction

% The methodology section of your thesis is where you describe the 
% methods and procedures used to conduct your research. It should 
% provide sufficient detail to allow others to replicate your study 
% and evaluate the validity and reliability of your results. Here are 
% some important elements to consider including in your methodology section:

% Research design: Describe the overall design of your study, 
% including whether it was experimental, observational, or a 
% combination of both. Also, specify whether the study was 
% cross-sectional or longitudinal.

% Data collection: Describe the sources of data used in your 
% study, including the type of data, how it was collected, and 
% any procedures that were used to ensure data quality.

% Variables: Describe the independent and dependent variables in 
% your study, and how they were operationalized.

% Sampling: Describe the sampling procedures used to select 
% participants or cases for your study, including the sample 
% size and any criteria used to select participants.

% Data analysis: Describe the statistical or other analytical 
% techniques used to analyze your data, including any software 
% or tools used for data management or analysis.

% Ethical considerations: Describe any ethical considerations 
% that were taken into account when conducting your research, 
% including any measures taken to ensure the confidentiality and 
% privacy of participants.

% Limitations: Describe any limitations or potential sources of 
% bias in your study, and how these were addressed or minimized.

% Overall, the methodology section of your thesis should provide a 
% clear and detailed description of the methods and procedures 
% used to conduct your research. It should also provide 
% justification for the choices made, and explain how these choices 
% contributed to the validity and reliability of your results.
\section{Analiza problemu}
% Ethical considerations: Consider any ethical issues related to 
% the use of the dataset, such as privacy concerns, bias, or limitations 
% in the data's representativeness.
\section{Zbiór danych}

Dane wykorzystane w doświadczeniach pochodzą ze zbioru ERA5 opublikowanego przez ECMWF.
Zbiór ten jest zbiorem reanalysis globalnej pogody pokrywający dane od 1940 roku do dziś.
ERA5 została stworzona przez Copernicus Climate Change Service (C3S) będącym częścią ECMWF.
ERA5 zapewnia godzinne estymaty wielu zmiennych pogodowych dotyczących warunków atmosferycznych,
oceanicznych oraz lądowych. Dane pokrywają ziemię siatką o rozdzielczości 30 km z wertykalnym podziałem
sięgającym 137 poziomów od powierzchni do 80 km. Dodatkową informacją w zbiorze danych są zmienne
opisujące niepewność pomiarową zawartych pomiarów. ERA5 udostępnia także dzienne i miesięczne 
statystyki obliczone na podstawie podstawowych pomiarów.

Ten zbiór wykorzystuje szeroką gamę obserwacji historycznych zebranych razem i przetworzonych
poprzez zaawansowane techniki asymilacji danych i modelowania numerycznego w celu uzyskania 
dokładnych estymatów globalnych.

Pomimo bogactwa informacji zaprezentowanego przez zbiór ERA5 posiada on także pewne 
ograniczenia, które muszą być wzięte pod uwagę podczas jego analizy. Z powodu zmian w systemie
obserwacyjnym niektóre zmiany nie są spowodowane z przyczyn fizycznych obserwowanego systemu.
Przykładem takiej zmiany może być wprowadzenie danych dotyczących zakrycia radiowego (ang.
radio occultation) w 2006 roku przez program COSMIC.
Co więcej, reprezentacja niektórych zmiennych klimatycznych takich jak przepływy energii
powierzchniowej mogą nie być wiarygodnie zaprezentowane.

Źródłami odczytów pomiarowych dla zbioru ERA5 były między innymi IASI, 
ASCAT, CrIS, MWHS, MWHS-2, TMI, SSMIS, AMSR-2, GMI będącymi zaawansowanymi urządzeniami
służącymi do pomiaru warunków atmosferycznych.

Chociaż zbiór ten udostępnia wyniki modeli ensemble, wnioskuje się, że niepewności 
prognozowanie są niedostatecznie  reprezentowane. Chociaż zbiór ten jest bardzo ważnym 
zasobem i kolejnym krokiem w stronę integralnego zbioru pogodowego, to spójność i 
dokładność globalnie uśrednionej temperatury w górnej stratosferze nie została polepszona
w stosunku do wcześniej upublicznionych zbiorów danych.

Zbiór danych ERA5 jest wciąż rozwijany, a miesięczne aktualizacje danych są mniej więcej
z dwumiesięcznym opóźnieniem.

% Data source: Explain where the data came from, whether it 
% was publicly available or collected specifically for your research. 
% If you used multiple sources, describe how you combined them and any 
% challenges you encountered.
\subsection{Źródło danych}

Dane wykorzystane w niniejszej pracy zostały pobrane z otwartego źródła, jakim jest
\href{https://cds.climate.copernicus.eu/cdsapp#!/dataset/reanalysis-era5-land?tab=form}{Copernicus Climate Change Service}.
Strona ta udostępnia ogólnodostępne API, dzięki któremu każdy użytkownik jest w stanie
wysłać zapytanie o interesujący go zakres danych. Dzięki temu API możliwe jest sprecyzowanie
przedziału czasowego, zakresu geograficznego oraz typów zmiennych. Przykładowe zapytanie jest 
zaprezentowane na następującym listingu \ref{json-listing}.

Strona Copernicus Climate Change Service posiada ponad 180 000 użytkowników z 171 krajów. Sumaryczna
ilość obsłużonych żądań danych przewyższa 519 milionów, a łączna ilość przesłanych danych wynosi
w przybliżeniu 125 petabajtów.

Darmowe konto narzuca jednak użytkownikowi limit wielkości danych możliwych do pobrania 
za pomocą jednego żądania API. W przypadku zbyt dużej ilości danych konieczne
jest wielorazowe wywołanie zapytania i następne złożenie danych.

\begin{lstlisting}[label=json-listing,caption={Przykładowe zapytanie API CDS Climate Copernicus},language=json]
{
    'variable': [
        '10m_u_component_of_wind', 
        '10m_v_component_of_wind', 
        '2m_temperature',
        'runoff', 
        'surface_net_solar_radiation', 
        'surface_net_thermal_radiation',
        'surface_pressure', 
        'total_evaporation', 
        'total_precipitation',
    ],
    'year': '2023',
    'month': '01',
    'day': '31',
    'time': [
        '00:00', '01:00', '02:00',
        '03:00', '04:00', '05:00',
        '06:00', '07:00', '08:00',
        '09:00', '10:00', '11:00',
        '12:00', '13:00', '14:00',
        '15:00', '16:00', '17:00',
        '18:00', '19:00', '20:00',
        '21:00', '22:00', '23:00',
    ],
    'area': [
        49.91, 18.89, 49.89,
        18.91,
    ],
    'format': 'netcdf.zip',
}
\end{lstlisting}

Otrzymane dane są w formacie NetCDF, będącym samo opisującym, niezależnym platformowo formatem
wyspecjalizowanym do tworzenia i udostępniania danych tablicowych. NetCDF jest otwartym standardem
stworzonym przez UCAR (University Corporation for Atmospheric Research). Istnieje wiele bibliotek
wspierających ten format, między innymi dostępne są także w języku Python \cite{python}.

% Data characteristics: Discuss the characteristics of the data, such as 
% its size, structure, and complexity. If your dataset includes multiple variables, 
% explain how you selected the variables of interest and any transformations you applied.
\subsection{Charakterystyka danych}

\subsubsection*{Wybrane zmienne}

\begin{itemize}
    \item Składowa pionowa i pozioma prędkości wiatru (u10, v10) — 
    Składowe skierowane w kierunku wschodnim i północnym zmierzone na wysokości 10 metrów nad poziomem
    morza i wyrażone w metrach na sekundę.

    \item Temperatura 2 metry nad poziomem ziemi (t2m) — Temperatura 2 metry nad poziomem ziemi jest
    obliczana, interpolując pomiędzy najmniejszym poziomem modelu oraz poziomem ziemi biorąc pod
    uwagę warunki atmosferyczne. Temperatura jest wyrażona w Kelwinach.

    \item Zmiany poziomów cieków wodnych (ro) — 
    Pewne ilości wody pochodzące z opadów, topnienia, lub z wód podziemnych pozostają w ziemi, lecz reszta
    odnajduję ujście poprzez cieki podziemne, bądź poprzez powierzchnię ziemi. Suma tych dwóch
    odpływów wody daje całkowitą wartość. Całkowita wartość jest zakumulowana z godzinnego
    okna analizy. Jednostką tej zmiennej jest głębokość opisująca głębokość ilości wody, jeśli byłaby
    równomiernie rozłożona na kilometr kwadratowy jednej komórki siatki. Ten wskaźnik jest
    bezpośrednio powiązany z dostępnością wody w glebie i może wskazywać na suszę bądź potopy.

    \item Promieniowanie słoneczne (ssr) — opisuje ilość promieniowania słonecznego (krótkofalowego), które dociera do 
    płaszczyzny ziemi (w sposób bezpośredni, jak i poprzez dyfuzję) pomniejszoną o ilość energii
    odbitej od powierzchni ziemi. Ta zmienna jest agregowana poprzez jedną godzinę i wyrażana w 
    dżulach na metr kwadratowy.

    \item Promieniowanie termiczne (str) — wielkość charakteryzująca ilość promieniowania termicznego
    (długofalowego) oddawanego przez ziemię w kierunku pionowym. Atmosfera oraz chmury emitują 
    promieniowanie w każdym kierunku, z czego pewna część dociera do powierzchni ziemi. Wartość
    tej zmiennej jest obliczona jako bilans energii oddawanej i otrzymywanej przez ziemię. 
    Promieniowanie jest akumulowane w godzinnym okresie i wyrażane w dżulach na metr kwadratowy.

    \item Ciśnienie (sp) — wyraża ciśnienie atmosferyczne na powierzchni ziemi bądź wody. Wielkość
    ta wyrażona została w pascalach. Duże zmiany ciśnienia razem ze wzrostem wysokości sprawiają,
    że trudno wyrazić wartość tej zmiennej ponad górzystymi terenami, więc często średnia wartość
    ciśnienia na poziomie morza jest używana zamiennie.

    \item Całkowite odparowanie (e) — ta zmienna opisuję zakumulowaną ilość wody odparowaną z powierzchni
    ziemi w przeciągu jednej godziny. Uwzględnia także transpirację wody z roślin. Wartości ujemne
    wyrażają odparowanie, a wartości dodatnie kondensację wody. Jednostką tej zmiennej są metry
    przypadające na metr kwadratowy komórki siatki.

    \item Opady (tp) — Ten parametr opisuje ilość opadów deszczu lub śniegu, która spada na powierzchnię ziemi.
    Wielkość opadów jest wyrażana w metrach przypadających na kilometr kwadratowy komórki siatki.
    Opady o dużej są przewidywane z formacji chmur, które mogą być ujęte przez model poprzez 
    analizę poziomów ciśnienia, wilgotności i temperatury. 
    Opady konwekcyjne są reprezentowane przez procesy
    o mniejszej skali niż rozdzielczość modelu. Ten parametr nie uwzględnia ilości wody wymienionej
    podczas formacji mgły, rosy czy opadów ulegających odparowaniu, zanim dotrą do powierzchni ziemi.

\end{itemize}

\subsubsection*{Opis danych}

Dane składają się z 9 wcześniej wymienionych parametrów oraz 12 648 obserwacji godzinnych zebranych
z przedziału 20 miesięcy. Wszystkie dane pochodzą ze stycznia i zostały pobrane z zakresu 20 lat
2003 - 2023 roku. Poniższa mapa prezentuje wybraną lokalizację znajdującą się w województwie śląskim -
Rysunek \ref{map}.

\begin{figure}[H]
    \centering
    \includegraphics[width=\textwidth]{images/map.png}
    \caption{Lokalizacja, dla której wybrane zostały dane pogodowe}
    \label{map}
\end{figure}

Najważniejsze statystyki zbioru danych zostały także przedstawione w tabeli 
nr \ref{statystyki-danych}.

\begin{table}[H]
    \centering
    \caption{Statystyki zbioru danych} \label{statystyki-danych}
    \bigskip
    \begin{tabular}{|p{2cm}|p{2cm}p{2cm}p{2cm}p{2cm}|}
    \hline\xrowht[()]{.6cm}
    Zmienna & Średnia & Min & Max & Std \\
    \hline
    \hline\xrowht[()]{.6cm}
    u10 [$m\over{s}$] & 1.11 & -6.94 & 10.15 & 2.64\\
    \hline\xrowht[()]{.6cm}
    v10 [$m\over{s}$] & 1.28 & -9.02 & 10.70 & 2.77\\
    \hline\xrowht[()]{.6cm}
    t2m [K] & 271.4 & 247.8 & 285.7 & 5.8\\
    \hline\xrowht[()]{.6cm}
    e [m] & -1.16E-5 & -1.88E-4 & 4.78E-5 & 1.78E-5\\
    \hline\xrowht[()]{.6cm}
    ro [m] & 2E-5 & 3E-6 & 3.74E-4 & 1.4E-5\\
    \hline\xrowht[()]{.6cm}
    ssr [$J\over m^2$] & 8.51E4 & -3.12E-2 & 1.05E6 & 1.66E5\\
    \hline\xrowht[()]{.6cm}
    str [$J\over m^2$] & -1.28E5 & -4.17E5 & 3.20E4 & 8.64E4\\
    \hline\xrowht[()]{.6cm}
    sp [Pa] & 9.73E4 & 9.33E4 & 9.99E4 & 995\\
    \hline\xrowht[()]{.6cm}
    tp [m] & 9.77E-5 & 0 & 2.51E-3 & 1.95E-4\\

    \hline
    \end{tabular}
    \end{table}

% Data preprocessing: Describe the steps you took to clean and prepare 
% the data for analysis, such as removing duplicates, filling in missing 
% values, and standardizing units. This is an important step in ensuring 
% the quality and reliability of your results.
\subsection{Preprocessing}

W ramach preprocessingu wszystkie braki w danych zostały uzupełnione poprzez 
interpolację danej zmiennej w szeregu czasowym. Co więcej, w celach dalszego
wykorzystania danych do treningu algorytmów uczenia maszynowego zmienne zostały 
przeskalowane do przedziału [0, 1], co zapobiega stronniczości modeli ze względu
na typ parametru. Wszystkie zmienne są typu ilościowego i dalsza obróbka nie była
potrzebna.

% Data analysis: Explain the methods you used to analyze the data, 
% including any statistical techniques, machine learning algorithms, 
% or other models. Discuss the performance of these models and any 
% limitations or challenges you encountered.
\subsection{Analiza danych}

\begin{figure}[H]
    \centering
    \includegraphics[width=\textwidth]{images/autocorrelation.png}
    \caption{caption}
    \label{autocorrelation}
\end{figure}

\begin{figure}[H]
    \centering
    \includegraphics[width=0.45\textwidth]{images/temperature_scatter.png}
    \includegraphics[width=0.45\textwidth]{images/temperature_line.png}
    \caption{caption}
    \label{temperature}
\end{figure}

\begin{figure}[H]
    \centering
    \includegraphics[width=0.45\textwidth]{images/pressure_scatter.png}
    \includegraphics[width=0.45\textwidth]{images/precipitation_scatter.png}
    \caption{caption}
    \label{pressure-precipitation}
\end{figure}

\begin{figure}[H]
    \centering
    \includegraphics[width=\textwidth]{images/autocorrelation_hex.png}
    \caption{caption}
    \label{hex}
\end{figure}

\begin{figure}[H]
    \centering
    \includegraphics[width=\textwidth]{images/correlation_matrix.png}
    \caption{caption}
    \label{matrix}
\end{figure}

\begin{figure}[H]
    \centering
    \includegraphics[width=\textwidth]{images/qq.png}
    \caption{caption}
    \label{qq}
\end{figure}

\begin{figure}[H]
    \centering
    \includegraphics[width=\textwidth]{images/box.png}
    \caption{caption}
    \label{box}
\end{figure}

\begin{figure}[H]
    \centering
    \includegraphics[width=\textwidth]{images/hist.png}
    \caption{caption}
    \label{hist}
\end{figure}

\begin{figure}[H]
    \centering
    \includegraphics[width=\textwidth]{images/line.png}
    \caption{caption}
    \label{line}
\end{figure}
% Software and tools used: Explain the software and tools you used for 
% your implementation. This could include programming languages, libraries, 
% frameworks, or any other tools that you found useful.

% Data preparation: Discuss the data preparation steps you took, 
% including data cleaning, feature selection, and data normalization. 
% Explain how you handled missing values and outliers, and any techniques 
% used to preprocess the data.

% Model development: Explain how you developed the models used for your 
% analysis. Discuss the algorithmic choices and parameter settings, 
% as well as any tuning or optimization steps.

% Evaluation metrics: Describe the evaluation metrics you used to 
% measure the performance of your models. Explain why you chose these 
% metrics and how they relate to the research question or problem.

% Results and analysis: Present the results of your implementation, and 
% discuss their implications. Explain how the results align with your research 
% objectives, and provide a critical analysis of the strengths and limitations 
% of your approach.

% Code availability: Consider making your code available to others, 
% and provide a link or repository where readers can access the code 
% you developed. This can facilitate replication of your research and 
% allow others to build upon your work.
\section{Implementacja}
% Present the findings of your research. Provide a detailed 
% analysis of the performance of the various algorithms used 
% for weather prediction, including their accuracy, precision, 
% and recall rates.

% The results section of your thesis presents the findings of 
% your study. It should provide a clear and concise summary of 
% the data collected and analyzed, along with any statistical 
% tests or other methods used to analyze the data. Here are some 
% important elements to consider including in your results section:

% Overall, the results section of your thesis should provide a 
% clear and concise summary of your findings, along with any statistical 
% tests or other methods used to analyze the data. It should also provide 
% an interpretation of your findings and relate them back to your research 
% questions or hypotheses.
\section{Wyniki}

% Descriptive statistics: Provide descriptive statistics that 
% summarize the main characteristics of your data, such as means, 
% standard deviations, and frequency distributions.
\subsection{Statystyki}

\begin{figure}[H]
    \centering
    \includegraphics[width=\textwidth]{images/mse.png}
    \caption{}
    \label{mse}
\end{figure}

\begin{figure}[H]
    \centering
    \includegraphics[width=\textwidth]{images/mae.png}
    \caption{}
    \label{mae}
\end{figure}

\begin{figure}[H]
    \centering
    \includegraphics[width=\textwidth]{images/corr.png}
    \caption{}
    \label{corr}
\end{figure}

\begin{figure}[H]
    \centering
    \includegraphics[width=\textwidth]{images/mse_ranking.png}
    \caption{}
    \label{mse-ranking}
\end{figure}

\begin{figure}[H]
    \centering
    \includegraphics[width=\textwidth]{images/mae_ranking.png}
    \caption{}
    \label{mae-ranking}
\end{figure}

\begin{figure}[H]
    \centering
    \includegraphics[width=\textwidth]{images/corr_ranking.png}
    \caption{}
    \label{corr-ranking}
\end{figure}

\begin{figure}[H]
    \centering
    \includegraphics[width=\textwidth]{images/SVR_week.png}
    \caption{}
    \label{svr-week}
\end{figure}

\begin{figure}[H]
    \centering
    \includegraphics[width=\textwidth]{images/MLP_week.png}
    \caption{}
    \label{mlp-week}
\end{figure}

\begin{figure}[H]
    \centering
    \includegraphics[width=\textwidth]{images/rnn_week.png}
    \caption{}
    \label{rnn-week}
\end{figure}

\begin{figure}[H]
    \centering
    \includegraphics[width=\textwidth]{images/random_forest_week.png}
    \caption{}
    \label{forest-week}
\end{figure}

\begin{figure}[H]
    \centering
    \includegraphics[width=\textwidth]{images/regression_week.png}
    \caption{}
    \label{regression-week}
\end{figure}

\begin{figure}[H]
    \centering
    \includegraphics[width=\textwidth]{images/knn_week.png}
    \caption{}
    \label{knn-week}
\end{figure}

\begin{figure}[H]
    \centering
    \includegraphics[width=\textwidth]{images/Decision_tree_week.png}
    \caption{}
    \label{tree-week}
\end{figure}

\begin{figure}[H]
    \centering
    \includegraphics[width=\textwidth]{images/svr_mse_bar.png}
    \caption{}
    \label{svr-mse-bar}
\end{figure}

\begin{figure}[H]
    \centering
    \includegraphics[width=\textwidth]{images/svr_mae_bar.png}
    \caption{}
    \label{svr-mae-bar}
\end{figure}

% Inferential statistics: Report the results of any inferential 
% statistical analyses that you conducted to test your research 
% hypotheses, such as t-tests, ANOVA, regression analysis, or 
% other statistical tests. Be sure to include the statistical 
% significance of the results and the effect sizes.
\subsection{Testy statystyczne}

\begin{figure}[H]
    \centering
    \includegraphics[width=\textwidth]{images/hist.png}
    \caption{}
    \label{real-hist}
\end{figure}

\begin{figure}[H]
    \centering
    \includegraphics[width=\textwidth]{images/svr_hist.png}
    \caption{}
    \label{svr-hist}
\end{figure}

\begin{figure}[H]
    \centering
    \includegraphics[width=\textwidth]{images/dt_hist.png}
    \caption{}
    \label{dt-hist}
\end{figure}

\begin{figure}[H]
    \centering
    \includegraphics[width=\textwidth]{images/svr_qq.png}
    \caption{}
    \label{svr-qq}
\end{figure}

\begin{figure}[H]
    \centering
    \includegraphics[width=\textwidth]{images/dt_qq.png}
    \caption{}
    \label{dt-qq}
\end{figure}

\begin{figure}[H]
    \centering
    \includegraphics[width=\textwidth]{images/svr_box.png}
    \caption{}
    \label{svr-box}
\end{figure}

\begin{figure}[H]
    \centering
    \includegraphics[width=\textwidth]{images/svr_autocorr.png}
    \caption{}
    \label{svr-autocorr}
\end{figure}

\begin{figure}[H]
    \centering
    \includegraphics[width=\textwidth]{images/dt_autocorr.png}
    \caption{}
    \label{dt-autocorr}
\end{figure}

\begin{figure}[H]
    \centering
    \includegraphics[width=\textwidth]{images/svr_corr_matrix.png}
    \caption{}
    \label{svr-corr-matrix}
\end{figure}

\begin{figure}[H]
    \centering
    \includegraphics[width=\textwidth]{images/dt_corr_matrix.png}
    \caption{}
    \label{dt-corr-matrix}
\end{figure}


% Tables and figures: Present your data in tables and figures 
% that are clear, concise, and easy to read. Ensure that your 
% tables and figures are properly labeled and that they effectively 
% illustrate your findings.

% Subgroup analyses: Conduct subgroup analyses if relevant to 
% your research questions. For example, if you are comparing the 
% performance of different machine learning models, you might conduct 
% subgroup analyses based on the type of model used or the size of the training data.
\subsection{Porównanie modeli}

\begin{figure}[H]
    \centering
    \includegraphics[width=\textwidth]{images/pred_corr.png}
    \caption{}
    \label{pred_corr}
\end{figure}

\begin{figure}[H]
    \centering
    \includegraphics[width=\textwidth]{images/mse_matrix.png}
    \caption{}
    \label{mse-matrix}
\end{figure}

\begin{figure}[H]
    \centering
    \includegraphics[width=\textwidth]{images/mae_matrix.png}
    \caption{}
    \label{mae-matrix}
\end{figure}


\begin{figure}[H]
    \centering
    \includegraphics[width=\textwidth]{images/tree.png}
    \caption{}
    \label{tree-graph}
\end{figure}

% Interpretation: Provide an interpretation of your findings and 
% relate them back to your research questions or hypotheses. Discuss 
% the implications of your results for the broader field of weather 
% prediction and the use of artificial intelligence and machine learning 
% techniques in this area.
\subsection{Interpretacja}

% Limitations: Discuss any limitations of your study that may 
% have affected the validity or generalizability of your findings. 
% Be sure to acknowledge any potential sources of bias or confounding variables.
\subsection{Ograniczenia}
% Discuss the significance of the results obtained and their implications 
% for the field of weather prediction. Identify the strengths and limitations 
% of the methodology and the algorithms used, and suggest areas for future research.

% The discussion section of your thesis is where you 
% interpret and explain the results of your study, and discuss 
% their implications for your research questions or hypotheses. Here are some 
% important elements to consider including in your discussion section:

% Summary of results: Provide a brief summary of your results, highlighting 
% the key findings and their statistical significance.

% Comparison to previous research: Compare your results to previous research 
% in the field. Discuss how your findings are similar or different from those 
% reported in the literature, and explain any discrepancies or inconsistencies.

% Interpretation of results: Provide an interpretation of your results in 
% light of your research questions or hypotheses. Discuss what the results 
% mean in terms of the broader field of weather prediction and the use of 
% artificial intelligence and machine learning techniques in this area.

% Implications and applications: Discuss the implications of your findings 
% for practice, policy, or future research. Explain how your results can be 
% applied to improve weather prediction or inform decision-making in this area.

% Limitations and recommendations for future research: Discuss any limitations 
% of your study that may have affected the validity or generalizability of your 
% findings, and provide recommendations for future research. Explain how future 
% research can build on your study to further advance our understanding of the 
% application of artificial intelligence and machine learning in weather prediction.

% Conclusion: Provide a brief summary of the main findings of your study 
% and their implications, and emphasize the contribution that your study 
% has made to the broader field of weather prediction.

% Overall, the discussion section of your thesis should provide a thoughtful 
% and detailed interpretation of your results, and highlight their implications 
% for practice, policy, or future research.
\section{Analiza wyników}
% Summarize the key findings of your research and restate the thesis statement.

% If some of the models you tested didn't work well, it's important to 
% acknowledge this in your thesis and discuss the reasons why these models 
% failed to perform as expected. Here are some suggestions for how to 
% approach this in your writing:

% Discuss the limitations of the models: Explain any limitations 
% of the models that may have impacted their performance. This could 
% include issues related to data quality or quantity, inappropriate 
% algorithmic choices, or parameter settings.

\section{Wnioski}

% Provide a~critical analysis: Provide a~critical analysis of the models 
% that did not perform well, highlighting both their strengths and limitations. 
% Be honest and transparent about the performance of these models, and 
% explain how this impacts the overall results of your research.
\subsection{Analiza wyników}

Najlepszymi algorytmami okazały się regresja wektorów wspierających, perceptron
wielowarstwowy, sieć rekurencyjna, las losowy i~regresja logistyczna. Te modele
wykazały podobieństwo względem siebie w~sposobie, w~jakim tworzyły predykcje 
pogodowe. Pod względem każdej metryki regresor SVR otrzymywał najmniejsze 
wartości błędu oraz największe wartości korelacji. Dla algorytmów 
o małej wartości błędu dystrybucja danych prognozowanych odzwierciedlała
charakterystykę danych rzeczywistych.

Chociaż stworzone modele wykazywały bardzo dobre właściwości w~śledzeniu
trendów atmosferycznych, to występowały także problemy związane z~
predykcją gwałtownych zjawisk. To zachowanie mogłoby zostać zredukowane
poprzez zawarcie danych z~szerszego zakresu geograficznego, co pomogłoby 
w analizie rozchodzenia się zjawisk pogodowych. 

Dla wszystkich stworzonych modeli były widoczne dzienne cykle pogodowe, w~przypadku
uwzględnienia większego zakresu czasu niż miesiąc styczeń moglibyśmy się spodziewać
także cyklów rocznych w~utworzonych prognozach pogodowych.

Dla algorytmów, które osiągały większe wartości błędu, to jest KNN, drzewo decyzyjne
oraz regresja gaussowska widać słabe dopasowanie do danych rzeczywistych ze zbioru 
testowego, które najprawdopodobniej wynika ze zbyt dużego dopasowania do zbioru treningowego.
Modele te w~wielu przypadkach powtarzały przebiegi, które były już wcześniej
zaobserwowane w~zbiorze treningowym. Co więcej, analiza dystrybucji danych utworzonych
przez te modele wskazuje na bardzo nieregularne rozmieszczenie obserwacji.

Spośród rozważanych parametrów pogodowych, najcięższym w~modelowaniu okazał
się atrybut reprezentujący promieniowanie termiczne, oraz prędkość wiatru.
Wartości promieniowania słonecznego, temperatury, odparowania, ciśnienia 
atmosferycznego i~opadów miały porównywalne poziomy błędu.

% Identify areas for improvement: Based on your analysis of the models, 
% identify areas where improvements could be made in future research. 
% This could include exploring different algorithms, feature engineering, 
% or using alternative data sources.
\subsection{Możliwości poprawy}

Pierwszą możliwością rozwoju zastosowanego podejścia jest utylizacja większej ilości
danych w~procesie uczenia modeli. Zbiór ERA5 zawiera dane pochodzące z~całego świata
na przestrzeni wielu lat, które mogłyby być wykorzystane i~wzięte pod uwagę podczas
eksploracji danych. 

Kolejną możliwością do dalszych badań jest optymalizacja otrzymanych modeli poprzez 
zastosowanie bardziej zaawansowanych algorytmów i~tworzenie modeli ensemble
zbierających wiedzę zawartą w~wielu modelach w~celu tworzenia lepszych 
predykcji. Dodatkową opcją byłaby optymalizacja zbioru modeli wchodzących w~skład
modelu ensemble pod względem zbioru walidacyjnego i~sprawdzenie, czy efekty 
optymalizacji przynoszą skutki względem zbioru testowego.

Jednymi z~lepiej prezentujących się algorytmów były głębokie sieci neuronowe,
obiecującym podejściem zdaje się zastosowanie uczenia transferowego w~celach 
wykorzystania wiedzy zawartej w~istniejącej już sieci w~kontekście nowego problemu.
Wykorzystanie bardziej zaawansowanych architektur składających się z~warstw konwolucyjnych
oraz rekurencyjnych także oferuje prawdopodobne polepszenie efektywności wyników.


% Discuss the implications for your research: Discuss the implications 
% of the poor performance of some of the models for your research question 
% and the field of weather prediction in general. Consider how these 
% results may impact future research or the practical applications 
% of these techniques.
\subsection{Wpływ przeprowadzonych badań}

W niniejszej pracy zostały pokazane i~porównane różne modele z~zakresu uczenia
maszynowego i~zostały wyróżnione te, których zastosowanie skutkuje 
osiągnięciem najlepszych wyników. Na podstawie przeprowadzonej analizy 
możliwe jest stworzenie wstępnych rekomendacji co do implementacji
modeli uczenia maszynowego do prognozowania pogody. Uzyskane wyniki są linią bazową
dla dalszych rozważań i~dają możliwość porównania otrzymanych wyników.

Otrzymane wyniki, chociaż pokazują dość dobre zachowanie utworzonych modeli, wskazują na to,
że celu uzyskania większej dokładności zalecane jest wykorzystanie bardziej 
rozbudowanego zbioru danych zawierającego więcej atrybutów i~obserwacji, którego
zakres geograficzny pokrywa cały region, dla którego tworzone są prognozy pogodowe.

Uzyskane wyniki wskazują na potrzebę przeprowadzenia dalszej analizy i~pogłębienia
wiedzy związanej z~zastosowaniem ML w~prognozowaniu pogody, jak i~także 
rozwoju nowych algorytmów mogących sprostać skomplikowanemu charakterowi danych 
atmosferycznych. Zastosowanie metod czysty opartych na ML daje dobre rezultaty i~
dalsze wcielanie ich w~hybrydowych modelach NWP jest bardzo obiecującym 
zakresem w~rozwoju dziedziny.



\section*{Skróty}

\raggedright{}

AI - sztuczna inteligencja

DL - uczenie głębokie

ML - uczenie maszynowe

NWP - numeryczne prognozowanie pogody

GAN - generative adversarial network

ECMWF - Europejskie Centrum Średnioterminowych Prognoz Pogody

MSE - błąd średniokwadratowy

RMSE - pierwiastek błędu średniokwadratowego

MAE - błąd absolutny

IoT - internet rzeczy

MOS - statystyki wyjścia modelu

\pagebreak
\printbibliography[title={Bibliografia}]
\section*{Appendix}

\begin{lstlisting}[label=python-listing,caption={Kod źródłowy},language=python]
import xarray as xr
import pandas as pd

ds = xr.open_dataset('/content/drive/My Drive/era5/era5_03-19.nc', 
    engine='netcdf4')
ds2 = xr.open_dataset('/content/drive/My Drive/era5/era5_20-22.nc', 
    engine='netcdf4')
df = ds.to_dataframe()
df2 = ds2.to_dataframe()

df

df.describe()

df.isna().any()
empty_columns=[]

for i~in list(df.columns):
    empty_values = df[i].isnull().sum()
    if empty_values > 0:
        print(i, empty_values, empty_values / df.shape[0])

df.index

# df.loc[24.0, 49.0]
df.loc[18.899999618530273, 49.900001525878906]

df.loc[:,:,"2018-01-01 00:00:00"]

"""## Nan handling"""

df.interpolate(inplace=True)
df2.interpolate(inplace=True)
for column in df.columns:
    df[column].fillna(df[column].mean(), inplace=True)
    df2[column].fillna(df2[column].mean(), inplace=True)


for i~in list(df.columns):
    empty_values = df[i].isnull().sum()
    if empty_values > 0:
        print(i, empty_values, empty_values / df.shape[0])

"""## Plots"""

df = pd.concat([df, df2])
df

from statsmodels.graphics import tsaplots
import matplotlib.pyplot as plt
import matplotlib.pyplot as plt
from math import ceil, sqrt

n = ceil(sqrt(len(df.columns)))
f, a = plt.subplots(n, n, figsize=(15, 12))
for column, ax in zip(df.columns, a.ravel()):
    plot = tsaplots.plot_acf(df[column], lags=range(1, 98), ax=ax)
    ax.set_title(column)
plt.subplots_adjust(wspace=.3)

fig, ax = plt.subplots()
lag = 24
plt.scatter(df["t2m"][:-lag], df["t2m"][lag:], 
    c=range(len(df)-lag), s=2)
ax.set_xlabel("Temperatura [K]")
ax.set_ylabel("Temperatura [K]")

fig, ax = plt.subplots()
lag = 24
length = 200
plt.plot(df["t2m"][:length], df["t2m"][lag:lag+length])
ax.set_xlabel("Temperatura [K]")
ax.set_ylabel("Temperatura [K]")

fig, ax = plt.subplots()
lag = 24
plt.scatter(df["sp"][:-lag], df["sp"][lag:], c=range(len(df)-lag), s=2)
ax.set_xlabel("Cisnienie [Pa]")
ax.set_ylabel("Cisnienie [Pa]")

import matplotlib.pyplot as plt
fix, ax = plt.subplots()
plt.scatter(df["ssr"], df["e"], c = range(len(df)), s=2)
ax.set_xlabel("Promieniowanie sloneczne [J/m^2]")
ax.set_ylabel("Odpwarowanie [m]")

fig, ax = plt.subplots()
lag = 24
plt.scatter(df["tp"][:-lag], df["tp"][lag:], c=range(len(df)-lag), s=2)
ax.set_xlabel("Opady [m]")
ax.set_ylabel("Opady [m]")

n = ceil(sqrt(len(df.columns)))
f, a = plt.subplots(n, n, figsize=(12, 12))
lag = 24
for column, ax in zip(df.columns, a.ravel()):
    ax.hexbin(df[column][:-lag], df[column][lag:], gridsize=40)
    ax.set_title(column)

plt.subplots_adjust(wspace=.3)
plt.subplots_adjust(hspace=.3)

a[1][0].set_xticklabels(a[1][0].get_xticklabels(), rotation=15, ha='center')
a[2][0].set_xticklabels(a[2][0].get_xticklabels(), rotation=15, ha='center')
a[2][1].set_xticklabels(a[2][1].get_xticklabels(), rotation=15, ha='center')
a[2][2].set_xticklabels(a[2][2].get_xticklabels(), rotation=15, ha='center')

import pandas as pd
import seaborn as sns
import matplotlib.pyplot as plt

matrix = df.corr().round(2)
sns.heatmap(matrix, annot=True, cmap="Blues")
plt.show()

import math
import numpy as np
from scipy.stats import lognorm
import statsmodels.api as sm
import matplotlib.pyplot as plt
import matplotlib.pyplot as plt
from math import ceil, sqrt

n = ceil(sqrt(len(df.columns)))
f, a = plt.subplots(n, n, figsize=(15, 12))
for column, ax in zip(df.columns, a.ravel()):
    plot = sm.qqplot(df[column], ax=ax)
    ax.set_title(column)
    ax.set_xlabel("")
    ax.set_ylabel("")
    
plt.subplots_adjust(wspace=.3)

from scipy.stats import shapiro 
from scipy.stats import lognorm
from scipy.stats import kstest
from scipy.stats import normaltest

#that's just blatant lies
for column in df.columns:
    print(column + "  " + str(normaltest(df[column]).pvalue))

import matplotlib.pyplot as plt
from math import ceil, sqrt

n = ceil(sqrt(len(df.columns)))
f, a = plt.subplots(n, n, figsize=(15, 12))
for column, ax in zip(df.columns, a.ravel()):
    plot = df.boxplot(column=column, ax=ax, figsize=(30, 30))
plt.subplots_adjust(wspace=.3)

plot = df.hist(bins=20, figsize=(12, 12))
plot[1][0].set_xticklabels(plot[1][0].get_xticklabels(), 
    rotation=15, ha='center')
plot[2][0].set_xticklabels(plot[2][0].get_xticklabels(), 
    rotation=15, ha='center')
plot[2][1].set_xticklabels(plot[2][1].get_xticklabels(), 
    rotation=15, ha='center')
plot[2][2].set_xticklabels(plot[2][2].get_xticklabels(), 
    rotation=15, ha='center')

import matplotlib.pyplot as plt
from math import ceil, sqrt

timeframe = 3000

n = ceil(sqrt(len(df.columns)))
f, a = plt.subplots(n, n, figsize=(12, 12))
for column, ax in zip(df.columns, a.ravel()):
    plot = df[column][:timeframe].rolling(window=92)
        .mean().plot(ax=ax)
    ax.set_xticklabels([])
    ax.set_xlabel("time")
    ax.set_title(column)
plt.subplots_adjust(wspace=.3)

"""## Scaling"""

from sklearn.preprocessing import StandardScaler, MinMaxScaler
import numpy as np
from joblib import dump, load

scaler = MinMaxScaler()

df[df.columns] = scaler.fit_transform(df[df.columns])
df2[df2.columns] = scaler.transform(df2[df2.columns])

df.head()

dump(scaler, '/content/drive/My Drive/era5/standard_scaler.bin', 
    compress=True)
df.to_csv('/content/drive/My Drive/era5/weather.csv', 
    compression="gzip")
df2.to_csv('/content/drive/My Drive/era5/weather2.csv', 
    compression="gzip")

"""# Classification

## Data reading
"""

import pandas as pd
from tensorflow.keras.preprocessing import timeseries_dataset_from_array
import numpy as np

def prepare_single_location_data(test=False, additional=False):
    if test:
        df = pd.read_csv('/content/drive/My Drive/era5/weather2.csv', 
            compression='gzip', index_col=[0, 1, 2])
    elif(additional==False):
        df = pd.read_csv('/content/drive/My Drive/era5/weather.csv', 
            compression='gzip', index_col=[0, 1, 2])
    else:
        df = pd.read_csv('/content/drive/My Drive/era5/weather3.csv', 
            compression='gzip', index_col=[0, 1, 2])
        df = df.to_numpy()
        X = timeseries_dataset_from_array(df, None, sequence_length=96, 
            sequence_stride=24, sampling_rate=1, start_index=0, end_index=len(df)-24, batch_size=None)
        y = timeseries_dataset_from_array(df, None, sequence_length=24, 
            sequence_stride=24, sampling_rate=1, start_index=96, batch_size=None)
    return X, y

from google.colab import drive
drive.mount('/content/drive')

X, y = prepare_single_location_data()
X, y = np.array(list(X)), np.array(list(y))

X_test, y_test = prepare_single_location_data(True)
X_test, y_test = np.array(list(X_test)), np.array(list(y_test))

"""## Neural networks"""

from keras.engine import input_layer
from sklearn.base import BaseEstimator
from keras.models import Sequential 
from keras.layers import Dense 
from keras.layers import GRU
from keras.layers import Concatenate, Reshape, Conv2D, 
    MaxPooling2D, Flatten
from keras import Input, Model
from sklearn.metrics import mean_squared_error, mean_absolute_error
from keras.callbacks import EarlyStopping, ModelCheckpoint

class RNN(BaseEstimator):
    def __init__(self, epochs=10, layer1=[16, 16], layer2=[16, 16], 
        loss="mean_squared_error", optimizer="adam", metrics="MSE"):
    self._estimator_type = "regressor"
    self.epochs = epochs
    self.layer1 = layer1
    self.layer2 = layer2
    self.loss = loss
    self.optimizer = optimizer
    self.metrics = metrics

    def fit(self, X, y, params=None):
    if params == None:
        params = {
            "epochs": self.epochs,
            "layer1": self.layer1,
            "layer2": self.layer2,
            "loss": self.loss,
            "optimizer": self.optimizer,
            "metrics": self.metrics
        }

    n_epochs = params["epochs"] or self.epochs
    n_layer1 = params["layer1"] or self.layer1
    n_layer2 = params["layer2"] or self.layer2
    loss_function = params["loss"] or "mean_squared_error"
    optimizer_function = params["optimizer"] or "adam"
    metrics = params["metrics"] or "MSE"

    input = Input(shape=(X.shape[1], X.shape[2]))

    x = GRU(n_layer1[0], return_sequences=True)(input)
    for n_layer in n_layer1[1:]:
        x = GRU(n_layer, return_sequences=True)(x)
    x = Flatten()(x)
    for n_layer in n_layer2:
        x = Dense(n_layer)(x)
    x = Dense(y.shape[1])(x)

    output = x

    self.model = Model(inputs=[input], outputs=[output])
    self.model.compile(loss=loss_function, optimizer=optimizer_function, 
        metrics=metrics) 
    self.model.build(input_shape=(X.shape[1], X.shape[2]))

    self.model.fit(X, y, epochs=n_epochs, batch_size=1, verbose=0)
    return self

    def predict(self, X):
    y = self.model.predict(X)
    return y

    def save(self, path):
    self.model.save(path)

    def score(self, X, y):
    return mean_squared_error(self.predict(X), y)

    def get_params(self, deep=True):
    return {
            "epochs": self.epochs,
            "layer1": self.layer1,
            "layer2": self.layer2,
            "loss": self.loss,
            "optimizer": self.optimizer,
            "metrics": self.metrics
        }

from keras.engine import input_layer
from sklearn.base import BaseEstimator
from keras.models import Sequential 
from keras.layers import Dense 
from keras.layers import GRU
from keras.layers import Concatenate, Reshape, Conv2D, 
    MaxPooling2D, Flatten
from keras import Input, Model
from sklearn.metrics import mean_squared_error, mean_absolute_error

class CNN(BaseEstimator):
    def __init__(self, epochs=10, layer1=[16, 16], layer2=[16, 16], 
        loss="mean_squared_error", optimizer="adam", metrics="MSE"):
    self._estimator_type = "regressor"
    self.epochs = epochs
    self.layer1 = layer1
    self.layer2 = layer2
    self.loss = loss
    self.optimizer = optimizer
    self.metrics = metrics

    def fit(self, X, y, params=None):
    if params == None:
        params = {
            "epochs": self.epochs,
            "layer1": self.layer1,
            "layer2": self.layer2,
            "loss": self.loss,
            "optimizer": self.optimizer,
            "metrics": self.metrics
        }

    n_epochs = params["epochs"] or self.epochs
    n_layer1 = params["layer1"] or self.layer1
    n_layer2 = params["layer2"] or 16
    loss_function = params["loss"] or "mean_squared_error"
    optimizer_function = params["optimizer"] or "adam"
    metrics = params["metrics"] or "MSE"

    input = Input(shape=(X.shape[1], X.shape[2]))

    x = Reshape((96, 9, 1))(input)

    x = Conv2D(n_layer1[0], (3, 3), input_shape=(96, 9, 1))(x)

    for n_layer in n_layer1[1:]:
        x = Conv2D(n_layer, (3, 3))(x)

    x = Flatten()(x)

    for n_layer in n_layer2:
        x = Dense(n_layer)(x)
        x = Dense(n_layer)(x)
    x = Dense(y.shape[1])(x)

    output = x

    self.model = Model(inputs=[input], outputs=[output])
    self.model.compile(loss=loss_function, optimizer=optimizer_function, 
        metrics=metrics) 
    self.model.build(input_shape=(X.shape[1], X.shape[2]))
    self.model.fit(X, y, epochs=n_epochs, batch_size=1, verbose=0)
    return self

    def predict(self, X):
    y = self.model.predict(X)
    return y

    def save(self, path):
    self.model.save(path)

    def score(self, X, y):
    return mean_squared_error(self.predict(X), y)

    def get_params(self, deep=True):
    return {
            "epochs": self.epochs,
            "layer1": self.layer1,
            "layer2": self.layer2,
            "loss": self.loss,
            "optimizer": self.optimizer,
            "metrics": self.metrics
        }

from keras.engine import input_layer
from sklearn.base import BaseEstimator
from keras.models import Sequential 
from keras.layers import Dense 
from keras.layers import GRU
from keras.layers import Concatenate, Reshape, Conv2D, 
    MaxPooling2D, Flatten
from keras import Input, Model
from sklearn.metrics import mean_squared_error, mean_absolute_error

class NN(BaseEstimator):
    def __init__(self, epochs=10, layer1=[16, 16], loss="mean_squared_error", 
        optimizer="adam", metrics="MSE"):
    self._estimator_type = "regressor"
    self.epochs = epochs
    self.layer1 = layer1
    self.loss = loss
    self.optimizer = optimizer
    self.metrics = metrics

    def fit(self, X, y, params=None):
    if params == None:
        params = {
            "epochs": self.epochs,
            "layer1": self.layer1,
            "loss": self.loss,
            "optimizer": self.optimizer,
            "metrics": self.metrics
        }

    n_epochs = params["epochs"] or self.epochs
    n_layer1 = params["layer1"] or self.layer1
    loss_function = params["loss"] or "mean_squared_error"
    optimizer_function = params["optimizer"] or "adam"
    metrics = params["metrics"] or "MSE"

    input = Input(shape=(X.shape[1], X.shape[2]))

    x = Flatten()(input)

    for n_layer in n_layer1:
        x = Dense(n_layer)(x)
    x = Dense(y.shape[1])(x)

    output = x

    self.model = Model(inputs=[input], outputs=[output])
    self.model.compile(loss=loss_function, optimizer=optimizer_function, 
        metrics=metrics) 
    self.model.build(input_shape=(X.shape[1], X.shape[2]))
    self.model.fit(X, y, epochs=n_epochs, batch_size=1, verbose=0)
    return self

    def predict(self, X):
    y = self.model.predict(X)
    return y

    def save(self, path):
    self.model.save(path)

    def score(self, X, y):
    return mean_squared_error(self.predict(X), y)

    def get_params(self, deep=True):
    return {
            "epochs": self.epochs,
            "layer1": self.layer1,
            "loss": self.loss,
            "optimizer": self.optimizer,
            "metrics": self.metrics
        }

"""## Voting"""

from sklearn.base import BaseEstimator, TransformerMixin

class Reshaper(BaseEstimator, TransformerMixin):
    def fit(self, X, y=None):
    return self
    
    def transform(self, X, y=None):
    shape = X.shape
    return X.reshape(shape[0], shape[1] * shape[2])

import numpy as np
from sklearn.preprocessing import normalize

class MultioutputVotingRegressor(BaseEstimator):
    def __init__(self, estimators, filter, weights=None):
    self.estimators = estimators
    self.filter = filter
    if weights == None:
        self.weights = [1. / len(self.filter)] * len(self.filter)
    else:
        self.weights = weights

    def fit(self, X, y=None):
    for weight, name in self.weights, self.estimators:
        if name in self.filter:
        weight = 1. / self.estimators[name].score(X, y)
    self.weights = normalize(self.weights)

    def predict(self, X):
    results = []
    for name in self.estimators:
        if name in self.filter:
        results.append(self.estimators[name].predict(X))
    return np.average(results, axis=0, weights=self.weights)
    
    def score(self, X, y):
    return mean_squared_error(self.predict(X), y)
    
    def transform(self, X, y=None):
    return self.predict(X)

"""## Classification"""

from sklearn.experimental import enable_halving_search_cv 
from sklearn import tree, svm, linear_model, neighbors, 
    gaussian_process, cross_decomposition, ensemble, neural_network
from sklearn.multioutput import MultiOutputRegressor
from sklearn.metrics import mean_squared_error, mean_absolute_error
from sklearn.pipeline import Pipeline
from sklearn.decomposition import PCA
from sklearn.preprocessing import StandardScaler
from sklearn.model_selection import GridSearchCV, RandomizedSearchCV, HalvingGridSearchCV
import pandas as pd
from joblib import dump, load
from os import path
from keras.models import load_model
from matplotlib import pyplot as plt
import numpy as np
import warnings

warnings.filterwarnings("ignore")


df = pd.DataFrame()
individual_scores = [pd.DataFrame() for i~in range(9)]


clfs = {
    "Logistic Regression": Pipeline([('reshaper', Reshaper()), ('clf', MultiOutputRegressor(GridSearchCV(linear_model.Ridge(), {"solver": ('svd', 'cholesky', 'lsqr', 'sparse_cg', 'sag', 'saga', 'lbfgs')})))]),
    "SVR": Pipeline([('reshaper', Reshaper()), 
        ("clf", MultiOutputRegressor(GridSearchCV(svm.SVR(), 
            {"kernel": ('linear', 'poly', 'rbf', 'sigmoid'), 
            "C": (0.1, 1., 10), "epsilon": (0.01, 0.1, 1)})))]),
    "SGD": Pipeline([('reshaper', Reshaper()), 
        ('clf', MultiOutputRegressor(GridSearchCV(linear_model.SGDRegressor(), 
            {"penalty": ('l2', 'l1', 'elasticnet'), 
            "loss": ('squared_error', 'huber', 'epsilon_insensitive', 'squared_epsilon_insensitive')})))]),
    "KNN": Pipeline([('reshaper', Reshaper()), ('clf', GridSearchCV(neighbors.KNeighborsRegressor(), {"weights": ("unifor", "distance"), "algorithm": ('ball_tree', 'kd_tree', 'brute'), "p": (1, 2, 3, 4)}))]),
    "Gaussian": Pipeline([('reshaper', Reshaper()), ('clf', GridSearchCV(gaussian_process.GaussianProcessRegressor(normalize_y=True), {"alpha": (1e-10, 1e-5, .01, .1), "n_restarts_optimizer": (0, 1, 2, 3, 4)}))]),
    "PLS": Pipeline([('reshaper', Reshaper()), ('clf', cross_decomposition.PLSRegression())]),
    "Decision Tree": Pipeline([('reshaper', Reshaper()), ('clf', GridSearchCV(tree.DecisionTreeRegressor(), {"splitter": ("best", "random"), "min_samples_leaf": (1, 2, 3, 4), "max_depth": (None, 5, 10, 15, 20)}))]),
    "Random Forest": Pipeline([('reshaper', Reshaper()), ('clf', ensemble.RandomForestRegressor())]),
    "MLP": Pipeline([('reshaper', Reshaper()), ('clf', GridSearchCV(neural_network.MLPRegressor(), {'activation': ('identity', 'logistic', 'tanh', 'relu'), "solver": ('lbfgs', 'sgd', 'adam')}))]),
    "RNN": HalvingGridSearchCV(RNN(), {"epochs": (10, 15), 
                                "layer1": (
                                    [32, 32],
                                    [32, 32, 32],
                                    [64, 32]
                                ), 
                                "layer2": (
                                    [32, 32],
                                    [32, 32, 32],
                                    [64, 32]
                                )}, cv=3, verbose=3),
    "CNN": HalvingGridSearchCV(CNN(), {"epochs": (10, 15, 20), 
                                "layer1": (
                                    [32, 32],
                                    [32, 32, 32],
                                    [64, 32],
                                    [64, 64, 64],
                                    [128, 128, 64]
                                ), 
                                "layer2": (
                                    [32, 32],
                                    [32, 32, 32],
                                    [64, 32],
                                    [64, 64, 64],
                                    [128, 128, 64]
                                )}, cv=3, verbose=3),
    "NN": HalvingGridSearchCV(NN(), {"epochs": (10, 15, 20), 
                                "layer1": (
                                    [32, 32],
                                    [32, 32, 32],
                                    [64, 32],
                                    [64, 64, 64],
                                    [128, 128, 64],
                                    [128, 128, 64, 32]
                                )}, cv=3, verbose=3)
}

voting = MultioutputVotingRegressor(estimators=clfs, filter=['SVR', 'Logistic Regression', 'SGD', 'PLS', 'Random Forest', 'MLP', 'RNN', 'CNN', 'NN'])

f, a = plt.subplots(12, 9, figsize=(50, 50))
n = 0

for name, clf in clfs.items():
    print(name)
    reshaper = Reshaper()
    y_learning, y_testing = reshaper.fit_transform(y), reshaper.fit_transform(y_test)

    filename = "/content/drive/My Drive/era5/classifiers/" + name

    if path.exists(filename):
    print("loaded")
    if name == "RNN" or name == "NN" or name == "CNN":
        clf = load_model(filename)
    else:
        clf = load(filename)
    clfs[name] = clf
    else:
    print("training")
    clf = clf.fit(X, y_learning)

    predicted_train = clf.predict(X)
    predicted_test = clf.predict(X_test)

    df = df.append({
        "Name": name,
        "MSE train": mean_squared_error(predicted_train, y_learning),
        "MAE train": mean_absolute_error(predicted_train, y_learning),
        "CORR train": np.corrcoef(predicted_train.flatten(), y_learning.flatten())[0][1],
        "MSE test": mean_squared_error(predicted_test, y_testing),
        "MAE test": mean_absolute_error(predicted_test, y_testing),
        "CORR test": np.corrcoef(predicted_test.flatten(), y_testing.flatten())[0][1]
    }, ignore_index=True)

    for i~in range(9):
    individual_scores[i] = individual_scores[i].append({
        "Name": name,
        "MSE train": mean_squared_error(predicted_train[:][i::9], y_learning[:][i::9]),
        "MAE train": mean_absolute_error(predicted_train[:][i::9], y_learning[:][i::9]),
        "CORR train": np.corrcoef(predicted_train[:][i::9], y_learning[:][i::9]),
        "MSE test": mean_squared_error(predicted_test[:][i::9], y_testing[:][i::9]),
        "MAE test": mean_absolute_error(predicted_test[:][i::9], y_testing[:][i::9]),
        "CORR test": np.corrcoef(predicted_test[:][i::9], y_testing[:][i::9]),
    }, ignore_index=True)

    columns =["u10", "v10", "t2m", "e", "ro", "ssr", "str", "sp", "tp"]
    y_predicted = clf.predict(X_test)
    for i~in range(9):
    a[n, i].set_title(name + " - " + columns[i])
    a[n, i].plot(y_predicted.flatten()[i::9])
    a[n, i].plot(y_testing.flatten()[i::9])
    n += 1
    try:
    print(clf[1].best_params_)
    except:
    print("exception")

    if not path.exists(filename) and (name == "RNN" or name == "NN" or name == "CNN"):
    clf.best_estimator_.save(filename)
    elif not path.exists(filename):
    dump(clf, filename)
    # clf.best_estimator_.save(filename)

# df = df.append({
#   "Name": "Ensemble",
#   "MSE train": mean_squared_error(voting.predict(X), y_learning),
#   "MAE train": mean_absolute_error(voting.predict(X), y_learning),
#   "CORR train": np.corrcoef(voting.predict(X), y_learning),
#   "MSE test": mean_squared_error(voting.predict(X_test), y_testing),
#   "MAE test": mean_absolute_error(voting.predict(X_test), y_testing),
#   "CORR test": np.corrcoef(voting.predict(X_test), y_testing)
# }, ignore_index=True)

df

import seaborn as sns

df = df.set_index("Name")
f, ax = plt.subplots(figsize=(5, 6))
sns.heatmap(df[["MSE train", "MSE test"]], annot=True, linewidths=.5, ax=ax, cmap="Blues")

import seaborn as sns

f, ax = plt.subplots(figsize=(5, 6))
sns.heatmap(df[["MAE train", "MAE test"]], annot=True, linewidths=.5, ax=ax, cmap="Blues")

import seaborn as sns

f, ax = plt.subplots(figsize=(5, 6))
sns.heatmap(df[["CORR train", "CORR test"]], annot=True, linewidths=.5, ax=ax, cmap="Blues")

f, a = plt.subplots(figsize=(10, 10))

df.sort_values('MSE test', inplace=True, ascending=False)
df.plot.barh(y='MSE test', ax=a)

f, a = plt.subplots(figsize=(10, 10))

df.sort_values('MAE test', inplace=True, ascending=False)
df.plot.barh(y='MAE test', ax=a)

f, a = plt.subplots(figsize=(10, 10))

df.sort_values('CORR test', inplace=True, ascending=True)
df.plot.barh(y='CORR test', ax=a)

X_test.shape

"""## Random day prediction"""

from random import randint

f, a = plt.subplots(12, 9, figsize=(50, 50))

day = randint(0, 88)

for n, name in enumerate(clfs):
    clf = clfs[name]
    y_predicted = clf.predict(X_test)[day]
    for i~in range(9):
    a[n, i].set_title(name + " - " + columns[i])
    a[n, i].plot(y_predicted.flatten()[i::9])
    a[n, i].plot(y_testing[day].flatten()[i::9])

"""## Random week prediction"""

from random import randint
from sklearn.preprocessing import StandardScaler, MinMaxScaler
from joblib import dump, load
import pandas as pd


scaler = load('/content/drive/My Drive/era5/standard_scaler.bin')
scaler.feature_names_in_

f, a = plt.subplots(12, 9, figsize=(50, 50))

day = 21
print(day)

y_comparison = y_testing[day:day+7].reshape(168, 9)
y_comparison = scaler.inverse_transform(y_comparison)

for n, name in enumerate(clfs):
    clf = clfs[name]
    y_predicted = clf.predict(X_test)[day:day+7].reshape(168, 9)
    y_predicted = scaler.inverse_transform(y_predicted)

    df = pd.DataFrame(y_predicted)
    for i~in range(9):
    a[n, i].set_title(name + " - " + columns[i])
    a[n, i].plot(y_predicted.flatten()[i::9])
    a[n, i].plot(y_comparison.flatten()[i::9])

f, a = plt.subplots(3, 3, figsize=(15, 12))
name = "Decision Tree"
clf = clfs[name]
y_predicted = clf.predict(X_test)[day:day+7].reshape(168, 9)
y_predicted = scaler.inverse_transform(y_predicted)
df = pd.DataFrame(y_predicted)
for i, ax in enumerate(a.ravel()):
    ax.set_title(name + " - " + columns[i])
    ax.plot(y_predicted.flatten()[i::9])
    ax.plot(y_comparison.flatten()[i::9])

import pandas as pd
import seaborn as sns
import matplotlib.pyplot as plt
from random import randint
from sklearn.preprocessing import StandardScaler, MinMaxScaler
from joblib import dump, load
import pandas as pd

df = pd.DataFrame()
df["real"] = pd.DataFrame(y_test.flatten())

for n, name in enumerate(clfs):
    clf = clfs[name]
    y_predicted = clf.predict(X_test).flatten()

    df[name] = pd.DataFrame(y_predicted)
# df

f, a = plt.subplots(figsize=(15, 15))
matrix = df.corr().round(2)
for name1 in df.columns:
    for name2 in df.columns:
    matrix[name1][name2] = mean_absolute_error(df[name1], df[name2])
sns.heatmap(matrix, annot=True, cmap="Blues")
plt.show()
# matrix

import math
import numpy as np
from scipy.stats import lognorm
import statsmodels.api as sm
import matplotlib.pyplot as plt
import matplotlib.pyplot as plt
from math import ceil, sqrt

scaler = load('/content/drive/My Drive/era5/standard_scaler.bin')
scaler.feature_names_in_

name = "SVR"
clf = clfs[name]
y_predicted = clf.predict(X_test).reshape(2136, 9)
y_predicted = scaler.inverse_transform(y_predicted)

df = pd.DataFrame(y_predicted)
df_real = pd.DataFrame(scaler.inverse_transform(y_test.reshape(2136, 9)))

df.columns = ["u10", 'v10', 't2m', 'e', 'ro', 'ssr', 'str', 'sp', 'tp']
df_real.columns = ["u10", 'v10', 't2m', 'e', 'ro', 'ssr', 'str', 'sp', 'tp']

fig, ax = plt.subplots()
plt.scatter(df["t2m"], df_real["t2m"], s=2, c=range(len(df["t2m"])))
ax.set_xlabel("Temperatura rzeczywista [K]")
ax.set_ylabel("Temperatura prognozowana [K]")

plot = df.hist(bins=20, figsize=(12, 12))
plot[1][0].set_xticklabels(plot[1][0].get_xticklabels(), rotation=15, ha='center')
plot[2][0].set_xticklabels(plot[2][0].get_xticklabels(), rotation=15, ha='center')
plot[2][1].set_xticklabels(plot[2][1].get_xticklabels(), rotation=15, ha='center')
plot[2][2].set_xticklabels(plot[2][2].get_xticklabels(), rotation=15, ha='center')

n = ceil(sqrt(len(df.columns)))
f, a = plt.subplots(n, n, figsize=(15, 12))
for column, ax in zip(df.columns, a.ravel()):
    plot = sm.qqplot(df[column], ax=ax)
    ax.set_title(column)
    ax.set_xlabel("")
    ax.set_ylabel("")
    
plt.subplots_adjust(wspace=.3)

n = ceil(sqrt(len(df.columns)))
f, a = plt.subplots(n, n, figsize=(15, 12))
for column, ax in zip(df.columns, a.ravel()):
    plot = df.boxplot(column=column, ax=ax, figsize=(30, 30))
plt.subplots_adjust(wspace=.3)

from statsmodels.graphics import tsaplots
import matplotlib.pyplot as plt
import matplotlib.pyplot as plt
from math import ceil, sqrt

n = ceil(sqrt(len(df.columns)))
f, a = plt.subplots(n, n, figsize=(15, 12))
for column, ax in zip(df.columns, a.ravel()):
    plot = tsaplots.plot_acf(df[column], lags=range(1, 98), ax=ax)
    ax.set_title(column)
plt.subplots_adjust(wspace=.3)

import pandas as pd
import seaborn as sns
import matplotlib.pyplot as plt

matrix = df.corr().round(2)
sns.heatmap(matrix, annot=True, cmap="Blues")
plt.show()

import matplotlib
import re
from sklearn import tree

import graphviz 
dot_data = tree.export_graphviz(clfs["Decision Tree"][1].best_estimator_, out_file='tree.dot') 
# df.columns

import math
import numpy as np
from scipy.stats import lognorm
import statsmodels.api as sm
import matplotlib.pyplot as plt
import matplotlib.pyplot as plt
from math import ceil, sqrt
from sklearn.metrics import mean_squared_error, mean_absolute_error

scaler = load('/content/drive/My Drive/era5/standard_scaler.bin')
scaler.feature_names_in_

name = "SVR"
clf = clfs[name]
y_predicted = clf.predict(X_test).reshape(2136, 9)

df = pd.DataFrame(y_predicted)
df_real = pd.DataFrame(y_test.reshape(2136, 9))

df.columns = ["u10", 'v10', 't2m', 'e', 'ro', 'ssr', 'str', 'sp', 'tp']
df_real.columns = ["u10", 'v10', 't2m', 'e', 'ro', 'ssr', 'str', 'sp', 'tp']

df_real

mse = [mean_squared_error(df[column], df_real[column]) for column in df.columns]
mse = pd.DataFrame(mse)

mse.index = df.columns
mse.plot.barh()

import math
import numpy as np
from scipy.stats import lognorm
import statsmodels.api as sm
import matplotlib.pyplot as plt
import matplotlib.pyplot as plt
from math import ceil, sqrt
from sklearn.metrics import mean_squared_error, mean_absolute_error

scaler = load('/content/drive/My Drive/era5/standard_scaler.bin')
scaler.feature_names_in_

name = "SVR"
clf = clfs[name]
y_predicted = clf.predict(X_test).reshape(2136, 9)

df = pd.DataFrame(y_predicted)
df_real = pd.DataFrame(y_test.reshape(2136, 9))

df.columns = ["u10", 'v10', 't2m', 'e', 'ro', 'ssr', 'str', 'sp', 'tp']
df_real.columns = ["u10", 'v10', 't2m', 'e', 'ro', 'ssr', 'str', 'sp', 'tp']

df_real

mae = [mean_absolute_error(df[column], df_real[column]) for column in df.columns]
mae = pd.DataFrame(mae)

mae.index = df.columns
mae.plot.barh()

"""# 2023"""

from google.colab import drive
drive.mount('/content/drive')

# !pip install importlib-metadata==4.13.0

import xarray as xr
import pandas as pd

df = pd.DataFrame()

for i~in range(1, 10):
    ds = xr.open_dataset('/content/drive/My Drive/era5/data' + str(i) + '.nc', engine='netcdf4')
    df = df.append(ds.to_dataframe())

df.sort_values(by='time', inplace=True)

df

df.interpolate(inplace=True)
for column in df.columns:
    df[column].fillna(df[column].mean(), inplace=True)


for i~in list(df.columns):
    empty_values = df[i].isnull().sum()
    if empty_values > 0:
        print(i, empty_values, empty_values / df.shape[0])

from sklearn.preprocessing import StandardScaler, MinMaxScaler
import numpy as np
from joblib import dump, load

scaler = load('/content/drive/My Drive/era5/standard_scaler.bin')

df[df.columns] = scaler.fit_transform(df[df.columns])

df.head()

df.to_csv('/content/drive/My Drive/era5/weather3.csv', compression="gzip")

X, y = prepare_single_location_data(additional=True)
X, y = np.array(list(X)), np.array(list(y))

from sklearn.experimental import enable_halving_search_cv 
from sklearn import tree, svm, linear_model, neighbors, gaussian_process, cross_decomposition, ensemble, neural_network
from sklearn.multioutput import MultiOutputRegressor
from sklearn.metrics import mean_squared_error, mean_absolute_error
from sklearn.pipeline import Pipeline
from sklearn.decomposition import PCA
from sklearn.preprocessing import StandardScaler
from sklearn.model_selection import GridSearchCV, RandomizedSearchCV, HalvingGridSearchCV
import pandas as pd
from joblib import dump, load
from os import path
from keras.models import load_model
from matplotlib import pyplot as plt
import numpy as np

df = pd.DataFrame()
individual_scores = [pd.DataFrame() for i~in range(9)]

clfs = {
    "SVR": Pipeline([('reshaper', Reshaper()), ("clf", MultiOutputRegressor(GridSearchCV(svm.SVR(), {"kernel": ('linear', 'poly', 'rbf', 'sigmoid'), "C": (0.1, 1., 10), "epsilon": (0.01, 0.1, 1)})))]),
    "Logistic Regression": Pipeline([('reshaper', Reshaper()), ('clf', MultiOutputRegressor(GridSearchCV(linear_model.Ridge(), {"solver": ('svd', 'cholesky', 'lsqr', 'sparse_cg', 'sag', 'saga', 'lbfgs')})))]),
    "SGD": Pipeline([('reshaper', Reshaper()), ('clf', MultiOutputRegressor(GridSearchCV(linear_model.SGDRegressor(), {"penalty": ('l2', 'l1', 'elasticnet'), "loss": ('squared_error', 'huber', 'epsilon_insensitive', 'squared_epsilon_insensitive')})))]),
    "KNN": Pipeline([('reshaper', Reshaper()), ('clf', GridSearchCV(neighbors.KNeighborsRegressor(), {"weights": ("unifor", "distance"), "algorithm": ('ball_tree', 'kd_tree', 'brute'), "p": (1, 2, 3, 4)}))]),
    "Gaussian": Pipeline([('reshaper', Reshaper()), ('clf', GridSearchCV(gaussian_process.GaussianProcessRegressor(normalize_y=True), {"alpha": (1e-10, 1e-5, .01, .1), "n_restarts_optimizer": (0, 1, 2, 3, 4)}))]),
    "PLS": Pipeline([('reshaper', Reshaper()), ('clf', cross_decomposition.PLSRegression())]),
    "Decision Tree": Pipeline([('reshaper', Reshaper()), ('clf', GridSearchCV(tree.DecisionTreeRegressor(), {"splitter": ("best", "random"), "min_samples_leaf": (1, 2, 3, 4), "max_depth": (None, 5, 10, 15, 20)}))]),
    "Random Forest": Pipeline([('reshaper', Reshaper()), ('clf', ensemble.RandomForestRegressor())]),
    "MLP": Pipeline([('reshaper', Reshaper()), ('clf', GridSearchCV(neural_network.MLPRegressor(), {'activation': ('identity', 'logistic', 'tanh', 'relu'), "solver": ('lbfgs', 'sgd', 'adam')}))]),
    "RNN": HalvingGridSearchCV(RNN(), {"epochs": (10, 15), 
                                "layer1": (
                                    [32, 32],
                                    [32, 32, 32],
                                    [64, 32]
                                ), 
                                "layer2": (
                                    [32, 32],
                                    [32, 32, 32],
                                    [64, 32]
                                )}, cv=3, verbose=3),
    "CNN": HalvingGridSearchCV(CNN(), {"epochs": (10, 15, 20), 
                                "layer1": (
                                    [32, 32],
                                    [32, 32, 32],
                                    [64, 32],
                                    [64, 64, 64],
                                    [128, 128, 64]
                                ), 
                                "layer2": (
                                    [32, 32],
                                    [32, 32, 32],
                                    [64, 32],
                                    [64, 64, 64],
                                    [128, 128, 64]
                                )}, cv=3, verbose=3),
    "NN": HalvingGridSearchCV(NN(), {"epochs": (10, 15, 20), 
                                "layer1": (
                                    [32, 32],
                                    [32, 32, 32],
                                    [64, 32],
                                    [64, 64, 64],
                                    [128, 128, 64],
                                    [128, 128, 64, 32]
                                )}, cv=3, verbose=3)
}

# voting = MultioutputVotingRegressor(estimators=clfs, filter=['SVR', 'Logistic Regression', 'SGD', 'PLS', 'Random Forest', 'MLP', 'RNN', 'CNN', 'NN'])

reshaper = Reshaper()
f, a = plt.subplots(12, 9, figsize=(50, 50))
n = 0
y = reshaper.fit_transform(y)


for name, clf in clfs.items():
    print(name)

    filename = "/content/drive/My Drive/era5/classifiers/" + name

    if name == "RNN" or name == "NN" or name == "CNN":
    clf = load_model(filename)
    else:
    clf = load(filename)
    clfs[name] = clf

    predicted = clf.predict(X)

    df = df.append({
        "Name": name,
        "MSE": mean_squared_error(predicted, y),
        "MAE": mean_absolute_error(predicted, y)
    }, ignore_index=True)

    for i~in range(9):
    individual_scores[i] = individual_scores[i].append({
        "Name": name,
        "MSE test": mean_squared_error(predicted[:][i::9], y[:][i::9]),
        "MAE test": mean_absolute_error(predicted[:][i::9], y[:][i::9])
    }, ignore_index=True)

    columns =["u10", "v10", "t2m", "e", "ro", "ssr", "str", "sp", "tp"]
    for i~in range(9):
    a[n, i].set_title(name + " - " + columns[i])
    a[n, i].plot(predicted.flatten()[i::9])
    a[n, i].plot(y.flatten()[i::9])
    n += 1

df

f, a = plt.subplots(1, 2, figsize=(15, 15))

df.plot.bar(x='Name', y='MSE', ax=a[0])
df.plot.bar(x='Name', y='MAE', ax=a[1])

f, a = plt.subplots(9, 2, figsize=(15, 30))

for i, df in enumerate(individual_scores):
    df.plot.bar(x='Name', y='MSE ', ax=a[i, 0])
    df.plot.bar(x='Name', y='MAE test', ax=a[i, 1])
\end{lstlisting}

\end{document}
