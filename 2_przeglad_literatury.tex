% Conduct a comprehensive review of the existing literature 
% on artificial intelligence and machine learning techniques 
% used in weather prediction. Discuss the strengths and limitations 
% of these approaches

% The literature review section of your thesis is a critical component 
% of your research, as it provides a comprehensive overview of the 
% current state of knowledge on the topic of weather prediction and 
% the application of artificial intelligence and machine learning 
% techniques in this area. Here are some important elements to consider 
% including in your literature review:

% Overall, your literature review should provide a comprehensive overview of 
% the existing literature on weather prediction and the application of artificial 
% intelligence and machine learning techniques in this area. It should also 
% highlight the importance of your research question, and explain how your research 
% contributes to the broader understanding of the field.
\section{Przegląd literatury}

Jako już dojrzała metodologia, zastosowanie uczenia maszynowego 
znajduje dużą duże odbicie w wielu artykułach i publikacjach naukowych. 
Wiele publikacji analizuje zastosowanie AI w hybrydowych jak i czystych 
algoritmach prognozowania pogody. 


% Background information: Begin by providing some background information on 
% the topic of weather prediction and the challenges that exist in this area. 
% This could include discussing the impact of weather on human life and the 
% economy, the limitations of current weather prediction methods, and the potential 
% benefits of using artificial intelligence and machine learning techniques to 
% improve accuracy.
\subsection{Tło teoretyczne}

\subsubsection{Metody NWP}

\subsubsection{Metody uczenia maszynowego}

% Overview of existing literature: Provide a comprehensive overview of the 
% existing literature on weather prediction and the use of artificial intelligence 
% and machine learning techniques in this area. This could involve summarizing the 
% findings of previous studies, identifying key themes and trends, and highlighting 
% any gaps or limitations in the current literature.
\subsection{Istniejące źródła}

Można znaleźć artykuły analizujące najczęściej występujące słowa
w publikacjach dotyczących algorytmów przewidywania pogody 
\cite{ml-in-weather-prediction}. Okazuje się że w artykułach traktujących o
metodach NWP najczęściej występującą frazą jest "prognozowanie wiatru". Innymi 
często występującymi sformuowaniami były "modele ensemble", "asymilacja danych",
"warunki ekstremalne". Pośród analizowanych artykułów słowo wiatr pojawiło się
ponad 200 razy, najczęściej w odniesieniu do źródeł odnawialnej energii i badań
w przewidywaniu siły wiatru. Słowo opady pojawiło się prawie 150 razy, zazwyczaj
odnosząc się do wykorzystania w prognozowaniu krótkoterminowym, post-processingu, czy
downscaling. 

Z kolei dla artykułów dotyczących uczenia maszynowego
w dziedzinie klimatu najczęściej występujące frazy to "zmiana klimatu", 
"wpływ na klimat", "warunki ekstremalne". Okazuje się że o wiele częstsze było
wystąpienie określenia "globalne modele klimatyczne" niż "regionalne modele 
klimatyczne". 

Wśród istniejących publikacji popularnymi tematami są także korekcja odchylenia 
temperatury i ciśnienia atmosferycznego, analiza promieniowania słonecznego w celach
zasilania instalacji fotowoltaicznych.

\subsubsection*{Wyniki analizowanych źródeł}

\Citeauthor*{machine-learning-for-applied-weather-prediction}
\cite{machine-learning-for-applied-weather-prediction} stworzyli system 
o nazwie DICast podnoszący dokładność modeli numerycznych o 10-15\% przy pomocy
uczenia maszynowego. Zaletą zastosowanego przez nich podejścia jest możliwość
użycia małego zbioru danych do treningu oraz dynamicznej aktualizacji do 
najnowszych informacji.

Badania podsumowane w  przez  \Citeauthor*{ai-revolutionises-weather-prediction}
\cite{ai-revolutionises-weather-prediction}
stwierdzają polepszenie dokładności metod numerycznych ulepszonych przy
pomocy ML w stosunku do bazowych modeli. Odnotowane polepszenie wyników
sięgało 12.4\% dla wilgotności, 5.2\% dla prędkości wiatru, 17.0\% dla
temperatury.

\Citeauthor*{weather-monitoring-using-artificial-intelligence}\cite{weather-monitoring-using-artificial-intelligence}
wybrali regresję liniową do modelowania pogody i zaobserwowali 3.5 procentową
dewiację dla prognozy na następny dzień.

Porównanie wielu modelów, w tym sieci neuronowej, sieci radialnej, 
drzew GBT, regresji liniowej i lasu losowego przez 
\Citeauthor*{developing-machine-learning-algorithms}
\cite{developing-machine-learning-algorithms} wskazuje na 
najlepsze wyniki uzyskiwane przez sieć neuronową. Osiągnęła ona 
współczynnik korelacji równy 84.62\% w zadaniu prognozowania
pogody z miesięcznym wyprzedzeniem. Także \Citeauthor*{weather-forecasting-using-dl} 
\cite{weather-forecasting-using-dl} wskazuje na dobrą dokładność
sieci neuronowych podczas prognozowania opadów.

\Citeauthor*{ml-applied-to-weather-forecasting}
\cite{ml-applied-to-weather-forecasting} wnioskuje że metody 
numeryczne są w stanie osiągnąć lepsze wyniki dla prognoz 
krótkoterminowych, lecz różnica dla prognoz długoterminowych
nie była już tak znaczna. Proponowanym wytłumaczeniem tego jest
brak stabilności modeli NWP i akumulacja błędów dla dłuższych 
okresów czasu. Z kolei uczenie maszynowe zdaje się być bardziej
odporne na perturbacje w warunkach początkowych i zachowuje 
stabilne wyniki nawet dla prognoz długoterminowych.

Mimo wszystko kwestia zastąpienia NWP przez metody uczenia maszynowego
wciąż nie jest roztrzygnięta jak wskazuje \Citeauthor*{can-dl-beat-numerical}
\cite{can-dl-beat-numerical}. Artykuł ten szacuje jakość prognoz 
deterministycznych publikowanych przez ECMWF na 80\% dokładności w przewidywaniu
poziomu ciśnienia na przełomie 7 dni. Z kolei temperatura może być 
prognozowana z błędem średnio-kwadratowym równym 2 stopnie. Wśród wymienionych
zastosowań uczenia głębokiego było szacowanie dystrybucji parametrów przy pomocy 
sieci rekurencyjnej i konwolucyjnej. Innym zastosowaniem było uchwycenie
niepewności z obserwacji w kontekście prognozowania opadów.

Jednym z bardziej wszechstronnych źródeł porównujących aż 24 modele z zakresu
uczenia maszynowego był \Citeauthor*{comparison-of-ml-methods}\cite{comparison-of-ml-methods}.
W tej pracy autor porównywał wybrane modele w celu przewidywania energii fotowoltaicznej
na następny dzień bazując na algorytmach NWP. W tym celu zastosowane było pięć metryk
weryfikacyjnych. Najlepszym modelem w przedstawionym problemie okazała się regresja 
kernel-ridge, wykazująca się także najdłuższym czasem treningu i wysokim zużyciem pamięci.
Algorytmem zalecanym do wykorzystania praktycznego był perceptron wielowarstwowy (MLP)
osiągający zbliżone wyniki do regresji kernel-ridge, lecz wymagający o wiele mniej zasobów.
Co więcej, jednym z wniosków tej pracy jest duże znaczenie wyboru zmiennych w danych wejściowych
na jakość wytrenowanego modelu. Powiększenie danych o dodatkowe informacje przyniosło
wyniki lepsze o 13.1\% w porównaniu do podstawowych zestawów atrybutów. Dalsze polepszenie
mogło być uzyskane poprzez dostrojenie hiperparametrów, umożliwiające zmniejszenie 
RMSE o 3.1\%. Dostosowanie hiperparametrów miało większy wpływ w algorytmach opartych
na drzewach decyzyjnych niż MLP, gradient boosting czy regresji. 

Kolejnym artykułem z tej dziedziny jest \Citeauthor*{coupling-data-science}\cite{coupling-data-science}
analizujący wartość nasłonecznienia. Głównym problemem napotkanym podczas
badań okazało się być znalezienie dobrej równowagi pomiędzy płynnością zmian a 
momentalnymi anomaliami w przewidywanych wartościach. Zastosowanie algorytmów
preferujących ciągłość kończyło się dążeniem prognozy do średniej wartości i 
dużymi błędami podczas zachmurzonych i słonecznych dni. Tak więc dalsza analiza 
wartości odstających i możliwości ich przewidywania jest potrzebna.

\Citeauthor{development-and-application-of-ml-in}\cite{development-and-application-of-ml-in}
zwraca uwagę na potrzebę dostosowania algorytmów AI do problemów związanych z prognozowaniem
pogody. Aspektem branym szczególnie pod uwagę w tej pracy jest wykorzystanie uczenia
maszynowego w celu przewidywania poziomu zanieczyszczeń powietrza. Większość
badań w celu dostosowania algorytmu do swoich danych stosuje dostrajanie hiperparametrów,
preprocessing lub wykorzystywanie modeli ensemble. 

Niektóre analizowane podejścia wykorzystują dość proste algorytmy w celu analizy
problemu. Tak na przykład \Citeauthor{weather-forecast-prediction-data-mining}
\cite{weather-forecast-prediction-data-mining} wykorzystuje drzewa decyzyjne C5 
aby prognozować temperaturę maksymalną, temperaturę minimalną, opady, prędkość wiatru
oraz odparowanie wody. Zastosowany algorytm wykorzystany został w celu tworzenia prognoz
na przekroju miesięcy i lat. Ilustruje to, że nawet proste modele AI są w stanie 
przynieść porządane wyniki, choć niekoniecznie mogące konkurować z dokładnością i 
złożonością algorytmów NWP.

Zastosowanie uczenia maszynowego w celach predykcji pogody znajduje zastosowanie
w sytuacjach w których ilość zasobów obliczeniowych jest ograniczona. Tak więc
\Citeauthor{weather-forecasting-using-ml}\cite{weather-forecasting-using-ml} wykorzystuje
mikrokontrolery przeprowadzające analizę w czasie rzeczywistym bazując na odczytach z 
czujników. Podobnie \Citeauthor{smart-weather-forecasting}\cite{smart-weather-forecasting}
proponuje wykorzystanie internetu rzeczy (IoT) aby zbierać i przetwarzać dane z różnych lokacji.
Dane gromadzone w czasie rzeczywistym mają potencjał rozszerzyć istniejące zbiory danych
i w konsekwencji polepszyć dokładność trenowanych modeli.

% Theoretical framework: Identify and discuss any theoretical frameworks that are 
% relevant to your research, and explain how these frameworks have been applied in 
% previous studies.
\subsection{Ramy teoretyczne}

% Methodological approaches: Discuss the different methodological approaches 
% that have been used in previous studies, including the types of data sources used, 
% the modeling techniques employed, and the evaluation metrics used to measure 
% performance.
\subsection{Podejście metodologiczne}

% Critical analysis: Provide a critical analysis of the existing literature, 
% highlighting any strengths or weaknesses in previous studies, and identifying 
% areas where further research is needed.
\subsection{Analiza krytyczna}

% The contribution of your research: Conclude by discussing how your research 
% contributes to the current state of knowledge on weather prediction and the 
% application of artificial intelligence and machine learning techniques in this area. 
% This could involve discussing the novelty of your approach, the potential 
% impact of your findings, or the implications for future research.
\subsection{Wkład pracy}