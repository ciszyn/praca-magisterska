% Conduct a comprehensive review of the existing literature 
% on artificial intelligence and machine learning techniques 
% used in weather prediction. Discuss the strengths and limitations 
% of these approaches

% The literature review section of your thesis is a critical component 
% of your research, as it provides a comprehensive overview of the 
% current state of knowledge on the topic of weather prediction and 
% the application of artificial intelligence and machine learning 
% techniques in this area. Here are some important elements to consider 
% including in your literature review:

% Overall, your literature review should provide a comprehensive overview of 
% the existing literature on weather prediction and the application of artificial 
% intelligence and machine learning techniques in this area. It should also 
% highlight the importance of your research question, and explain how your research 
% contributes to the broader understanding of the field.
\section{Przegląd literatury}

% Background information: Begin by providing some background information on 
% the topic of weather prediction and the challenges that exist in this area. 
% This could include discussing the impact of weather on human life and the 
% economy, the limitations of current weather prediction methods, and the potential 
% benefits of using artificial intelligence and machine learning techniques to 
% improve accuracy.
\subsection{Tło teoretyczne}

% Overview of existing literature: Provide a comprehensive overview of the 
% existing literature on weather prediction and the use of artificial intelligence 
% and machine learning techniques in this area. This could involve summarizing the 
% findings of previous studies, identifying key themes and trends, and highlighting 
% any gaps or limitations in the current literature.
\subsection{Istniejące źródła}

% Theoretical framework: Identify and discuss any theoretical frameworks that are 
% relevant to your research, and explain how these frameworks have been applied in 
% previous studies.
\subsection{Ramy teoretyczne}

% Methodological approaches: Discuss the different methodological approaches 
% that have been used in previous studies, including the types of data sources used, 
% the modeling techniques employed, and the evaluation metrics used to measure 
% performance.
\subsection{Podejście metodologiczne}

% Critical analysis: Provide a critical analysis of the existing literature, 
% highlighting any strengths or weaknesses in previous studies, and identifying 
% areas where further research is needed.
\subsection{Analiza krytyczna}

% The contribution of your research: Conclude by discussing how your research 
% contributes to the current state of knowledge on weather prediction and the 
% application of artificial intelligence and machine learning techniques in this area. 
% This could involve discussing the novelty of your approach, the potential 
% impact of your findings, or the implications for future research.
\subsection{Wkład pracy}