% Provide an overview of the topic, 
% motivation for the research, and the scope of the study.

% The introduction to your thesis is an essential 
% part of your work, as it provides the reader with an 
% overview of your research, its significance, and its objectives. 
% Here are some important elements to consider including in your introduction:

% Overall, your introduction should provide the reader with a clear 
% understanding of the research problem you are addressing, the methods 
% you have used to explore this problem, and the significance of your 
% findings. It should also engage the reader and motivate them to read 
% on, by highlighting the importance and relevance of your research.

% "The results of this research demonstrate that the application 
% of artificial intelligence and machine learning techniques in 
% weather prediction is a promising approach, as the results 
% obtained are comparable to those achieved by established techniques in the field."

% In this thesis statement, you highlight the contribution 
% of your research in showing the promise of artificial 
% intelligence and machine learning techniques in weather 
% prediction. You also emphasize that your results are 
% comparable to those obtained by other approaches, which 
% underscores the reliability and validity of your research findings.
\section{Wstęp}

Sztuczna inteligencja (AI) jako gwałtownie rozwijająca się 
dziedziną informatyki znajduję coraz więcej zastosowań w 
problemach modelujących zachowanie środowiska które nas otacza. 
Nowe algorytmy oraz metody stosowane w zakresie AI mają coraz
większy wpływ na społeczeństwo oraz na sposób w jaki interpretujemy,
przetwarzamy i wykorzystujemy dostępne dane. Wykorzystanie metod
sztucznej inteligencji pozwala na przeanalizowanie dużych ilości
danych, oraz na odnajdowanie i wzięcie pod uwagę skomplikowanych związków
i informacji w nich ukrytych. Możliwość zastosowania metod AI względem 
szerokiego zakresu typu danych, oraz wszechstronność algorytmów jest jednym z 
powodów które umożliwiają zastosowanie ich w celach prognozowania oraz modelowania
pogody. Rosnące zapotrzebowania na dokładne oraz długoterminowe przewidywanie
w zakresie monitorowania klimatu, wieloletnich oraz krótkoterminowych prognoz
pogody skłania do wykorzystywania coraz to nowych algorytmów i podejść
do problemu przewidywania pogody. Wykorzystanie AI może skutkować jednocześnie 
zwiększeniem efektywności w wykorzystaniu zasobów obliczeniowych, zmniejszeniem
ilości wkładu ludzkiego oraz zwiększeniem jakości prognoz.

Jedną z metod która szczególnie w ostatnich latach zyskała popularność w zakresie
AI jest uczenie głębokie (DL). Zastosowanie DL poprzez uczenie sieci neuronowych
znajduje szerokie zastosowanie w rozpoznawaniu i procesowaniu obrazów, wideo oraz
mowy. Wykorzystaniu propagacji wstecznej w algorytmie uczenia sieci neuronowych
pozwala na szybkie i mniej obciążające stworzenie modelu, przynajmniej w porównaniu
z tradycyjnymi metodami numerycznymi wykorzystywanymi w celach przewidywaniu pogody.

Rozwijające się zbiory danych opisujące parametry opisujące pogodę 
pozwalają na wykorzystywanie algorytmów i metod z zakresu uczenia maszynowego (ML)
oraz AI. W ostatnich latach można zaobserwować wzrost wykorzystania tych metod
przez organizacje klimatyczne oraz coraz więcej badań związanych z wykorzystaniem
AI w numerycznym prognozowaniu pogody, jak i zupełnym zastąpieniu numerycznych
prognoz pogodowych (NWP) poprzez 
AI.

% Background and context: Begin by providing some background 
% information on the topic of weather prediction and the application 
% of artificial intelligence and machine learning techniques in this area. 
% This could include a brief overview of the current state of the field, the 
% challenges that exist, and the potential benefits of using these techniques.
\subsection{Kontekst i zastosowania}

Monitorowanie i przewidywanie pogody znajduje zastosowanie w dziedzinach takich
jak agrokultura, kontrolowanie stanu zanieczyszczeń powietrza, 
przewidywanie warunków drogowych w celu zwiększenia bezpieczeństwa drogowego,
przewidywanie rozprzestrzeniania się pożarów naturalnych,
przewidywanie katastrof naturalnych, zastosowanie odnawialnych źródeł energi,
oraz wiele innych. Dokładne przewidywanie warunków pogodowych ma krytyczne znaczenie
dla prawidłowego funkcjonowania wielu organizacji. Amerykańskie centra
informacji o środowisku szacują wielkość zniszczeń wynikających z działania
pagody w 2015 roku na 7.9 miliardów dolarów\cite{using-artificial-intelligence-to-improve}.
Z drugiej strony możliwości prognozowania nasłonecznienia dają szansę na zaoszczędzenie
455 milionów dolarów do 2040 w ramach wykorzystania odnawialnych źródeł energii.

Porozumienie paryskie jest prawnie wiążącą umową międzynarodową,
której celem jest ograniczenie globalnego ocieplenia o 1.5$^{\circ}$C
do 2050 roku w porównaniu do poziomów przedindustrialnych. Wykorzystanie
uczenia maszynowego mogłoby pozwolić na bardziej przemyślane zarządzanie
gospodarką energetyczną oraz wykorzystanie źródeł energii odnawialnych.

Dotychczas najczęściej stosowanym podejściem do celów prognozowania
pogody są algorytmy numeryczne bazujące na rozwiązywaniu równań różniczkowych
charakteryzujących zachowanie atmosfery i pogody. Skomplikowana natura
problemu oraz fakt że równania opisujące dynamikę cieczy oraz termodynamikę
cechują się chaotycznością sprawiają że numeryczne symulacje komputerowe
nie zawsze osiągają wymaganą dokładność, a ich złożoność obliczeniowa
wymaga dużych klastrów komputerowych w celu osiągnięcia celu.

Chociaż duże sieci neuronowe wykorzystywane do prognozowania pogody
mogą zawierać kilka milionów parametrów, co jest porównywalne z modelami
NWP\cite{can-dl-beat-numerical}, to wykorzystanie takiego modelu już po traningu
jest mało wymagająca obliczeniowo i można się spodziewać o wiele mniejszych wymagań
obliczeniowych do wykorzystywania takiego modelu. Co więcej, maksymalna rozdzielczość
modeli NWP stosowanych do globalnych prognoz nie przekracza 5km, 
co może skutkować pominięciem lokalnych zjawisk. Algorytmy AI dają możliwość
parametryzowania i brania pod uwagę zjawisk o charakterze lokalnym.

W ostatnich latach zaobserwowano wzrastającą tendencę do stosowania hybrydowych
algorytmów bazujących na modelach numerycznych, ale także uczeniu maszynowym
wspomagającym w inicjalizacji modelu, parametryzacji modelu, korygowaniu
wyników generowanych przez model oraz szacowaniu pewności wygenerowanej
predykcji. Zastosowanie modeli bazujących czysto na wykorzystywaniu uczenia
maszynowego dotychczas wykazuje gorszą dokładność od podejść hybrydowych.
Największym problemem w tworzeniu modeli AI w celu predykcji pogody jest
zachowanie ograniczeń fizycznych, które nie są bezpośrednio częścią modelu,
w przeciwieństwie do algorytmów NWP.

Wykorzystanie uczenia głębokiego wiążę się z możliwością zastosowania 
zaawansowanych algorytmów które znalazły już zastosowanie w innych dziedzinach.
Sieci neuronowe umożliwiłiby wykorzystanie Transfer Learning, umożliwiający
wydobycie ze zbioru danych podstawowych relacji i charakterystyk,


% Problem statement: Clearly articulate the research problem you are 
% addressing, and explain why it is important. This could involve discussing 
% the limitations of current weather prediction methods, or the potential 
% benefits of improving the accuracy and reliability of these predictions.
\subsection{Sformuowanie problemu}

% Research objectives: Clearly state the objectives of your research, 
% including any hypotheses you are testing or research questions you 
% are exploring. This helps the reader understand the scope and focus of your work.
\subsection{Cele pracy}

% Research methods: Provide an overview of the research methods you have used, 
% including the data sources you have used, the models you have developed, 
% and the evaluation metrics you have used to measure performance.
\subsection{Metody badania}

% Contribution to the field: Conclude by discussing the contribution 
% of your research to the field of weather prediction and the broader 
% field of artificial intelligence and machine learning. This could 
% include highlighting the novelty of your approach, the potential 
% impact of your findings, or the implications for future research.
\subsection{Wkład w dziedzinę}