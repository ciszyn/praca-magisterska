% Provide an overview of the topic, 
% motivation for the research, and the scope of the study.

% The introduction to your thesis is an essential 
% part of your work, as it provides the reader with an 
% overview of your research, its significance, and its objectives. 
% Here are some important elements to consider including in your introduction:

% Overall, your introduction should provide the reader with a clear 
% understanding of the research problem you are addressing, the methods 
% you have used to explore this problem, and the significance of your 
% findings. It should also engage the reader and motivate them to read 
% on, by highlighting the importance and relevance of your research.

% "The results of this research demonstrate that the application 
% of artificial intelligence and machine learning techniques in 
% weather prediction is a promising approach, as the results 
% obtained are comparable to those achieved by established techniques in the field."

% In this thesis statement, you highlight the contribution 
% of your research in showing the promise of artificial 
% intelligence and machine learning techniques in weather 
% prediction. You also emphasize that your results are 
% comparable to those obtained by other approaches, which 
% underscores the reliability and validity of your research findings.
\section{Wstęp}

% Background and context: Begin by providing some background 
% information on the topic of weather prediction and the application 
% of artificial intelligence and machine learning techniques in this area. 
% This could include a brief overview of the current state of the field, the 
% challenges that exist, and the potential benefits of using these techniques.
\subsection{Kontekst}

% Problem statement: Clearly articulate the research problem you are 
% addressing, and explain why it is important. This could involve discussing 
% the limitations of current weather prediction methods, or the potential 
% benefits of improving the accuracy and reliability of these predictions.
\subsection{Sformuowanie problemu}

% Research objectives: Clearly state the objectives of your research, 
% including any hypotheses you are testing or research questions you 
% are exploring. This helps the reader understand the scope and focus of your work.
\subsection{Cele pracy}

% Research methods: Provide an overview of the research methods you have used, 
% including the data sources you have used, the models you have developed, 
% and the evaluation metrics you have used to measure performance.
\subsection{Metody badania}

% Contribution to the field: Conclude by discussing the contribution 
% of your research to the field of weather prediction and the broader 
% field of artificial intelligence and machine learning. This could 
% include highlighting the novelty of your approach, the potential 
% impact of your findings, or the implications for future research.
\subsection{Wkład w dziedzinę}