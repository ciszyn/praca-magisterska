% Summarize the key findings of your research and restate the thesis statement.

% If some of the models you tested didn't work well, it's important to 
% acknowledge this in your thesis and discuss the reasons why these models 
% failed to perform as expected. Here are some suggestions for how to 
% approach this in your writing:

% Discuss the limitations of the models: Explain any limitations 
% of the models that may have impacted their performance. This could 
% include issues related to data quality or quantity, inappropriate 
% algorithmic choices, or parameter settings.

\section{Wnioski}

% Provide a critical analysis: Provide a critical analysis of the models 
% that did not perform well, highlighting both their strengths and limitations. 
% Be honest and transparent about the performance of these models, and 
% explain how this impacts the overall results of your research.
\subsection{Analiza wyników}

Najlepszymi algorytmami okazały się regresja wektorów wspierających, perceptron
wielowarstwowy, sieć rekurencyjna, las losowy i regresja logistyczna. Te modele
wykazały podobieństwo względem siebie w sposobie, w jakim tworzyły predykcje 
pogodowe. Pod względem każdej metryki regresor SVR otrzymywał najmniejsze 
wartości błędu oraz największe wartości korelacji. Dla algorytmów 
o małej wartości błędu dystrybucja danych prognozowanych odzwierciedlała
charakterystykę danych rzeczywistych.

Chociaż stworzone modele wykazywały bardzo dobre właściwości w śledzeniu
trendów atmosferycznych, to występowały także problemy związane z 
predykcją gwałtownych zjawisk. To zachowanie mogłoby zostać zredukowane
poprzez zawarcie danych z szerszego zakresu geograficznego, co pomogłoby 
w analizie rozchodzenia się zjawisk pogodowych. 

Dla wszystkich stworzonych modeli były widoczne dzienne cykle pogodowe, w przypadku
uwzględnienia większego zakresu czasu niż miesiąc styczeń moglibyśmy się spodziewać
także cyklów rocznych w utworzonych prognozach pogodowych.

Dla algorytmów, które osiągały większe wartości błędu, to jest KNN, drzewo decyzyjne
oraz regresja gausowska widać słabe dopasowanie do danych rzeczywistych ze zbioru 
testowego, które najprawdopodobniej wynika ze zbyt dużego dopasowania do zbioru treningowego.
Modele te w wielu przypadkach powtarzały przebiegi, które były już wcześniej
zaobserwowane w zbiorze treningowym. Co więcej, analiza dystrybucji danych utworzonych
przez te modele wskazuje na bardzo nieregularne rozmieszczenie obserwacji.

Spośród rozważanych parametrów pogodowych, najcięższym w modelowaniu okazał
się atrybut reprezentujący promieniowanie termiczne, oraz prędkość wiatru.
Wartości promieniowania słonecznego, temperatury, odparowania, ciśnienia 
atmosferycznego i opadów miały porównywalne poziomy błędu.

% Identify areas for improvement: Based on your analysis of the models, 
% identify areas where improvements could be made in future research. 
% This could include exploring different algorithms, feature engineering, 
% or using alternative data sources.
\subsection{Możliwości poprawy}

Pierwszą możliwością rozwoju zastosowanego podejścia jest utylizacja większej ilości
danych w procesie uczenia modeli. Zbiór ERA5 zawiera dane pochodzące z całego świata
na przestrzeni wielu lat, które mogłyby być wykorzystane i wzięte pod uwagę podczas
eksploracji danych. 

Kolejną możliwością do dalszych badań jest optymalizacja otrzymanych modeli poprzez 
zastosowanie bardziej zaawansowanych algorytmów i tworzenie modeli ensemble
zbierających wiedzę zawartą w wielu modelach w celu tworzenia lepszych 
predykcji. Dodatkową opcją byłaby optymalizacja zbioru modeli wchodzących w skład
modelu ensemble pod względem zbioru walidacyjnego i sprawdzenie, czy efekty 
optymalizacji przynoszą skutki względem zbioru testowego.

Jednymi z lepiej prezentujących się algorytmów były głębokie sieci neuronowe,
obiecującym podejściem zdaje się zastosowanie uczenia transferowego w celach 
wykorzystania wiedzy zawartej w istniejącej już sieci w kontekście nowego problemu.
Wykorzystanie bardziej zaawansowanych architektur składających się z warstw konwolucyjnych
oraz rekurencyjnych także oferuje prawdopodobne polepszenie efektywności wyników.


% Discuss the implications for your research: Discuss the implications 
% of the poor performance of some of the models for your research question 
% and the field of weather prediction in general. Consider how these 
% results may impact future research or the practical applications 
% of these techniques.
\subsection{Wpływ przeprowadzonych badań}

W niniejszej pracy zostały pokazane i porównane różne modele z zakresu uczenia
maszynowego i zostały wyróżnione te, których zastosowanie skutkuje 
osiągnięciem najlepszych wyników. Na podstawie przeprowadzonej analizy 
możliwe jest stworzenie wstępnych rekomendacji co do implementacji
modeli uczenia maszynowego do prognozowania pogody. Uzyskane wyniki są linią bazową
dla dalszych rozważań i dają możliwość porównania otrzymanych wyników.

Otrzymane wyniki, chociaż pokazują dość dobre zachowanie utworzonych modeli, wskazują na to,
że celu uzyskania większej dokładności zalecane jest wykorzystanie bardziej 
rozbudowanego zbioru danych zawierającego więcej atrybutów i obserwacji, którego
zakres geograficzny pokrywa cały region, dla którego tworzone są prognozy pogodowe.

Uzyskane wyniki wskazują na potrzebę przeprowadzenia dalszej analizy i pogłębienia
wiedzy związanej z zastosowaniem ML w prognozowaniu pogody, jak i także 
rozwoju nowych algorytmów mogących sprostać skomplikowanemu charakterowi danych 
atmosferycznych. Zastosowanie metod czysty opartych na ML daje dobre rezultaty i 
dalsze wcielanie ich w hybrydowych modelach NWP jest bardzo obiecującym 
zakresem w rozwoju dziedziny.

