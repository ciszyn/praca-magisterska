Following work concerns analysis of use of artificial intelligence in meteorological applications for weather forecasting problems. In order to interpret effect of appropriate models, dataset ERA5 created by European Centre for Medium-Range Weather Forecasts was utilized. This datasource contains data from global measurements assimilated by newest numerical models. Analysis was focused on short term weather forecast in time range of 24 hours, and geographical range of one location. Created models were optimized by hiperparameter tuning.
Among considered models were SVR, logistic regression, SGD, K-nearest neighbors, gaussian regression, decision tree, random forest, neural networks and PLS decomposition.
Among considered metrics were mean squared error, mean absolute error and correlation factor. Additionally, statistical properties of generated forecasts in comparison to real observations were taken into account. As a result of conducted experiments, a comprehensive comparison of analysed models was created, and obtained conclusions underlined advantages and disadvantages of appropriate methods.
Additional emphasis was put on analysis of machine learning methods in hybrid configurations with numerical models of weather simulation.